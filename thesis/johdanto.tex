\chapter{Johdanto}
\label{ch:johdanto}
Miehittämättömät ilma-alukset, eli englanniksi Unmanned Aerial Vehicle:t (UAV)
tai tuttavallisemmin ``dronet'', ovat kokoajan lisääntyvä tekniikan alue.
Asevoimat käyttävät droneja laajasti, joko tiedustelussa tai
taistelukentillä, mutta niille löytyy lisääntyvässä määrin käyttöä myös
siviilimaailmassa. Esimerkiksi Amazon suunnittelee käyttävänsä niitä tavaran
nopeassa ja halvassa kuljetuksessa, ja video- ja valokuvauksessa ne ovat
mahdollistaneet halvat ilmakuvat. Myös tutkimuksessa ja muussa datan
keräyksessä UAV:t ja muut miehittämättömät ajoneuvot tulevat helpottamaan
ihmisten työtä, ja avaruustutkimuksessa miehittämättömät alukset ovat jo
pitkään tehneet työtä, joka on ollut liian kallista tai vaarallista ihmisille.
UAV:den laajoihin käyttöalueisiin kuuluvat myös telekommunikaatioalan
sovellukset, joissa ne ovat halvempia ja joustavampia kuin satelliitit.

Kuvissa~\ref{fig:small_12dollar_quad} ja~\ref{fig:big_predator} näkyy hyvin,
kuinka erilaisia dronet voivat olla; Toinen on 12 dollarin arvoinen,
leikkikäyttöön tarkoitettu, kädenkokoinen malli, ja toinen on päälle 4:n
miljoonan dollarin arvoinen sotilastiedustelumalli, jonka siipiväli on 14,8 m.

\begin{figure}[H]
\centering
\begin{minipage}{.5\textwidth}
  \centering
  \includegraphics[width=0.8\textwidth]{small_12dollar_quad}
  \caption{Leikkikäyttöön tarkoitettu drone~\cite{miniQuad}}
\label{fig:small_12dollar_quad}
\end{minipage}%
\begin{minipage}{.5\textwidth}
  \centering
  \includegraphics[width=0.8\textwidth]{big_predator}
  \caption{Sotilastiedustelumalli, General Atomics MQ-1
    Predator~\cite{bigPredator}}
\label{fig:big_predator}
\end{minipage}
\end{figure}

UAV:t koostuvat monista eri osista. Korkealla tasolla UAV voidaan esimerkiksi
jakaa seuraaviin osiin:
\begin{itemize}

\item teholähde, yleensä akku, mutta voi kehittyneissä malleissa olla myös
esimerkiksi aurinkopaneeli.

\item sulautetut järjestelmät. UAV:n ohjaus, anturidatan keräys ja käsittely,
ja mahdollisesti myös tekoäly pohjautuvat nykyaikana halpaan ja helposti
saataavaan laskentatehoon.

\item anturit. UAV:den pääkäyttö on nykyaikana kaikenlaisen datan kerääminen,
joten anturit ovat melkein kaikkissa droneissa olennaisia.

\item aktuaattorit. Kaikki dronet tarvitsevat ainakin ohjausaktuaattoreita.
Multiroottoreissa (useamman roottorin avulla lentävä drone) tämä tarkoittaa
ainoastaan moottoreiden nopeuden säätöä, mutta lentokonemalleissa tarvitaan myös
siivekkeiden ohjausta.

\item kommunikaatiojärjestelmät. Harrastemalleissa on yleensä yksinkertaisia,
lyhytkantoisia radiojärjestelmiä, joiden tarkoitus on yleensä vain ohjata
dronea. Sotilasmalleissa taas on tarpeen välittää paljon dataa kumpaankin
suuntaan, myös silloin, kun ohjaaja ja alus ovat eri mantereilla.
\end{itemize}

Kuvassa~\ref{fig:harrastedrone} on kuvattu yksinkertaisen harrastemallin
elektroniikkaa ja lohkojen välisiä kytkentöjä. Kyseessä on
multiroottorimalli, joka on harrastajien keskuudessa yleinen sen
yksinkertaisuudesta johtuen. Kuvasta nähdään, että kyseisessä mallissa on
aktuaattoreina neljä moottoria ja neljä nopeudensäädintä. Dronesta löytyy
radiovastaanotin ja lentämistä helpottava lentotietokone. Teholähteenä toimii
yksi litiumioniakku.
\begin{figure}[H]
  \begin{center}
    \includegraphics[width=1.0\textwidth]{harrastedrone}
  \end{center}
  \caption{Harrastedronen elektroniset järjestelmät, perustuu kuvaan lähteessä~\cite{RedditQuad}}
\label{fig:harrastedrone}
\end{figure}


Tämän työn tarkoituksena on käsitellä edellä listattuja UAV:n elektronisia osia
tarkemmin.
Luvussa 2 esitellään, mitä eri teholähteitä UAV:t voivat käyttää.
Luvussa 3 käsitellään dronen sulautettuja järjestelmiä, kuten
suunnistusjärjestelmiä tai automaattiohjausta.
Luvussa 4 käsitellään droneissa käytettyjä antureita ja luvussa 5 niissä
käytettyjä aktuaattoreita.
Luvussa 6 käydään läpi UAV:n ohjauksessa ja tiedonsiirrossa käytettäviä
kommunikaatiojärjestelmiä.
Luvussa 7 on yhteenveto työn sisällöstä.
