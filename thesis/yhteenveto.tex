\chapter{Yhteenveto}
\label{ch:yhteenveto}

Dronejen elektroniset järjestelmät ovat monimutkainen kokonaisuus. Jokaisen
järjestelmän osan suunnittelu riippuu dronen toiminnallisista vaatimuksista,
muista järjestelmän osista ja tietenkin käytössä olevista resursseista.

Akkujen energiatiheyden paraneminen mahdollistaa yhä pienemmät dronet. Suurempi
energiatiheys tarkoittaa, että dronet pystyvät toimimaan pidempään, tai että ne
voivat olla yhä pienempiä. Myös uusiutuvat energianlähteet, energiansieppaus ja
muut vaihtoehdot, kuten polttokennot, parantavat dronejen toiminta-aikaa,
-kantamaa ja mahdollistavat niiden käytön myös tilanteissa jossa resurssit ovat
vähissä. Kaikki kehitys teholähteiden parissa on tärkeää myös ympäristön
kannalta.

Pienentyneet laskentakomponentit mahdollistavat yhä itsenäisempiä UAV:ta. Myös
niiden hinnan lasku on mahdollistanut halvat dronet harraste- ja
tutkimuskäytössä. Lentoa helpottavat autopilottijärjestelmät mahdollistavat
myös nykyään suositut quad-roottorit, joiden ohjaus ilman autopilottia olisi
todella haastavaa.  Laskentakomponenttien tehosta saadaan nykyaikaisella
tekoälyllä myös yhä enemmän irti. Ilmasta otettuja kuvia voidaan analysoida
konenäön avulla, ja näiden analyysien perusteella voidaan tehdä päätöksiä
koneoppimista hyödyntäen.

Dronen antureiden avulla voidaan kerätä monenlaista dataa. Voidaan ottaa
ilmakuvia viihdekäyttöä varten, käyttää infrapunakuvia peltojen tai eläinten
tarkkailua varten, tai mallintaa ympäristöä LIDAR:n avulla. GPS ja IMU piirit
mahdollistavat dronen paikannuksen ja ohjauksen, vaikka näköyhteyttä ei
olisikaan.

Sähkömoottorit ovat pakollinen osa pieniä droneja, joihin polttomoottori olisi
liian iso. Sähkömoottoreilla tuotetaan työtövoimaa, ja quad-roottoreiden
tapauksessa niitä voi myös käyttää ohjaukseen. Servomoottoreilla voidaan ohjata
kiinteäsiipisten dronejen ohjauspintoja.

Dronen kommunikaatiojärjestelmät mahdollistavat tietenkin niiden
kauko-ohjauksen, mutta niillä voi tehdä myös muuta; Niiden avulla dronet voivat
välittää tietoa, keskustella toistensa kanssa ja muodostaa wlan-verkkoja
alueille joissa niitä tarvitaan, esimerkiksi kriisitapauksissa.

Nykyaikaset dronet ovat seurausta kehityksestä jokaisella dronen elektroniikan
osa-alueella, ja jatkokehitys vaatii myös jatkuvaa kehitystä jokaisella
alueella.
