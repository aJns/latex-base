\hyphenation{tuu-li-ener-gia}

\chapter{Teholähde}
\label{ch:teholahde}

Tässä luvussa käsitellään UAV:den käyttämiä teholähteitä, ja niihin kohdistuvia
erityisvaatimuksia. Useimmissa UAV:ssa moottorit käyvät koko ajan, joka luo
jatkuvan energian tarpeen. Tämän lisäksi anturit ja aktuaattorit voivat myös
luoda piikkejä energian tarpeeseen.

Perinteisesti UAV:t ovat käyttäneet teholähteenään joko akkuja tai
polttomoottoreita. Uudet teholähteet tekevät kumminkin tuloaan.  Esimerkiksi
aurinkovoima soveltuu hyvin varsinkin korkealla toimiviin droneihin, koska
pilvet eivät vaikuta aurinkokennojen tehokkuuteen. Aurinkoenergian lisäksi
ympäristöä voidaan hyödyntää energiasieppausmetodein, esimerkiksi muuntamalla
UAV:n potkurista syntyvää värinää sähköenergiaksi, tai hyödyntämällä
toiminta-alueen ilmavirtauksia liikkumiseen. Myös polttokennot ovat
mahdollinen teholähde, vaikkakin nykyisellään niiden matala energiatiheys
rajoittaa niiden käyttöä lennokeissa~\cite{Bradley2007}.

Hybriditeholähteet ovat lupaavimpia tulevaisuuden teholähteitä, koska
polttomoottoria lukuunottamatta tässä tekstissä käsitellyt teholähteet eivät
kykene tuottamaan tehoa tarpeeksi paljon, tasaisesti tai ennustettavasti, tai
jos pystyvät, ne rajoittavat UAV:n toimintaa liikaa. Polttomoottorin ongelmana
taas on sen aiheuttama saaste, ja joissakin tilanteissa sen aiheuttama melu tai
lämpö voi olla epäedullista, vaikkapa jos halutaan estää dronen havaitseminen.
``Hybridi'' tarkoittaa, että teholähde koostuu useammasta teholähteestä, joka
tarkoittaa yleensä akkua, ja jotain, joka lataa sitä, esimerkiksi
aurinkokennoa.  Jotkut hybriditeholähteet koostuvat jopa kolmesta eri
teholähteestä: Akusta, polttokennosta ja aurinkokennosta.~\cite{Chen2010}

\section{Akut ja polttomoottorit}
Käytännössä kaikki dronet tarvitsevat akun riippumatta siitä, onko se ainoa
teholähde vai ei. Monet tutkimuksen kohteena olevista teholähteistä tuottavat
tehoa melko katkonaisesti. Osa, esimerkiksi energiansieppaus potkurin
värinästä, ei edes pysty huolehtimaan tehontuotannosta kokonaan, vaan niiden
tehtävä on pitkittää toiminta-aikaa. Myös polttomoottoria teholähteenään
käyttävät dronet hyötyvät akusta, koska se mahdollistaa elektroniikan
toiminnan vaikka moottori sammuisikin. Akun avulla sammunut moottori voitaisiin
myös pyrkiä käynnistämään uudelleen ilman ulkopuolista apua.

Litiumioniakkuja on perinteisesti käytetty kulutuselektroniikassa, ja myös
harrastedronet kuuluvat tähän ryhmään. Litiumioniakut ovat kuitenkin
yleistyneet myös isommissa sähköajoneuvoissa, koska niillä on monia etuja
perinteisiin lyijyakkuihin verrattuna. Niiden energiatiheys on korkeampi,
ne ovat pitkäkestoisia, niiden itsepurkautuvuus on vähäistä, ja niillä ei ole
muisti-ilmiötä. Akkuja, joissa esiintyy muisti-ilmiötä, täytyy purkaa tyhjäksi
säännöllisesti, tai sen kapasiteetti alenee. Litiumioniakut ovat kuitenkin
muita akkutyyppejä kalliimpia johtuen litiumin
harvinaisuudesta.~\cite{Sparacino2012}

Polttomoottoreita käytetään yleensä isommissa UAV:ssa, kuten sotilasmalleissa.
Esimerkiksi Yhdysvaltojen ilmavoimien käyttämä MQ-1B Predator käyttää Rotax
914F polttomoottoria~\cite{usaf-predator}, joka tuottaa parhaimmillaan
84,5 kW tehoa~\cite{rotax-914}. Ilmavoimien
nettisivujen~\cite{usaf-predator} mukaan Predatorin kantama on 770 mailia, eli
noin 1239 kilometriä.

\sloppy{Suurin syy, miksi isommat UAV:t käyttävät polttomoottoreita akkujen
  sijaan, on polttoaineiden huomattavasti suurempi energiatiheys. Esimerkiksi
  Panasonicin NCR18650B~\cite{panasonic-ncr18650b} litiumioniakun energiatiheys
  on noin 0.87 MJ/kg, kun taas yleisesti käytössä olevan lentokonepolttoaineen
  AVGAS 100LL~\cite{shell-avgas}, jota myös Predator käyttää, energiatiheys on
  noin 44 MJ/kg, joka on 50 kertainen verrattuna litiumioniakkuun.
}



\section{Polttokennot teholähteenä}
Polttokenno on laite joka muuntaa kemiallista energiaa sähköenergiaksi.
Kemiallinen energia syntyy kennoon syötetystä vetypitoisesta polttoaineesta
ja hapesta, tai muusta oksidoivasta aineesta, joiden välinen sähkökemiallinen
reaktio vapauttaa energiaa.

Droneissa polttokennoja käytetään yleensä hybriditeholähteissä, joissa
on sekä polttokenno että akku. Näin teholähde pystyy vastaamaan esimerkiksi
nousun vaatimiin tehopiikkeihin. Kun UAV on lentokorkeudella ja tehonkulutus
matalaa ja tasaista, polttokenno voi ladata akkua. Haittana
hybriditeholähteessä on teholähteen monimutkaisempi rakenne.~\cite{Gao2005}

Lähteessä~\cite{Bradley2007} käsitellään polttokennoa käyttävän UAV-prototyypin
ominaisuuksia ja suorituskykyä. Kyseinen protyyppi käyttää ainoastaan
polttokennoa tehonlähteenään. Polttokenno huipputeho on 465 W ja sen
energiatiheys on noin 0.187 MJ/kg. Prototyyppia on testattu noin 110 s
lennoilla, joiden aikana se on lentänyt noin 1200~m matkan.  Näiden testien
perusteella UAV:n lentoajaksi arvioidaan maksimissaan noin 43 minuuttia.

\section{Aurinkokennot ja energiansieppaus}
Aurinkokennoja ja energiansieppausta voidaan myös hyödyntää
hybriditeholähteissä pidentämään UAV:den lentoaikoja. Energiansieppaus
tarkoittaa ympäristössä ilmenevän energian muuttamista käytettävään
sähköenergiaan~\cite{Anton2011}. Aurinko- ja tuulienergia ovat jo melko
jokapäiväisiä energiansieppausmenetelmiä, mutta esimerkiksi myös ihmisen
ruumiinlämpöä tai moottorin värinää voidaan muuntaa sähköenergiaksi.

Aurinkoenergian ja muiden UAV:n ulkopuolisiin ilmiöihin perustuvien
energiansieppaustapojen tehontuotantoa on kuitenkin vaikea varmistaa, koska
niihin ei pysty vaikuttamaan juuri muuten kuin UAV:n toiminta-alueen
valinnalla. Akulla toimintavarmuutta voidaan parantaa vähentämällä UAV:n
riippuvuutta ulkoisista ilmiöistä.  Niiden järkevät käyttöalueet ja -tilanteet
ovat joka tapauksessa rajattuja; Päiväntasaajalla aurinkoenergia on
varteenotettava vaihtoehto, Suomessa ei niinkään.  Parhaassakin tapauksessa
tehonkulutusta pitää suunnitella tarkaan.  Lähteessä~\cite{Jaw-KuenShiau2009}
käsitellyn prototyypin 3:n aurinkopaneelin yhteen laskettu huipputeho on noin
72 W. Aurinkokennojen tuottama teho riippuu kuitenkin niiden jännitteestä,
joten ne käytännössä vaativat maksimitehopisteen seurantajärjestelmän (Maximum
Power Point Tracking, MPPT). Myös akun lataaminen vaatii lisää elektroniikkaa,
koska aurinkokennojen jännitetaso on yleensä liian matala, jolloin tarvitaan
hakkureita nostamaan jännite käytettävälle tasolle.

Energiaa voidaan siepata esimerkiksi dronen omasta värinästä~\cite{Priya2009}
tai ympäröivistä sääilmiöistä. Värinä on siepattavan energian lähteenä melko
luotettava verrattuna sääilmiöihin, jotka riippuvat muun muassa sijainnista,
vuoden- ja vuorokaudenajasta sekä lentokorkeudesta. Hyvä esimerkki sääilmiön
hyödyntämisestä on ilmavirtausten käyttäminen UAV:n korkeuden lisäämiseen
moottorin sijaan~\cite{Cutler2010}.

\section{Langaton tehonsiirto}
Langattomalla tehonsiirrolla voidaan poistaa tarve laskeutua akun latausta
varten, mikä helpottaa latausta varsinkin malleissa, jotka tarvitsevat
kiitoradan. Latausasemia voidaan sijoittaa maastoon joihin laskeutuminen ei
olisi mahdollista, jolloin UAV pystyy toimimaan kyseisellä alueella paljon
pidempään.

Lähteessä~\cite{Griffin2012} käsitellään magneettiseen resonanssiin perustuvaa
langatonta tehonsiirtoa. Kyseisessä artikkelissa ideana on siirtää energiaa
UAV:n avulla maassa sijaitsevalle anturille, mutta kyseistä menetelmää
voidaan myös käyttää toisinpäin lataamaan UAV:ta. Suurimpana ongelmana
kyseisessä menetelmässä on UAV:n asennon ja sijainnin pitäminen optimaalisena,
mutta siitä huolimatta menetelmä saavutti lähes 5 W jatkuvan tehonsiirron.

Toinen langattoman tehonsiirron teknologia perustuu laseriin.
Lähteessä~\cite{Achtelik2011} on esitetty järjestelmä, jossa tukiasema tähtää
lasersäteen aurinkokennoon, joka on liitetty droneen. Kyseisellä järjestelmällä
on saavutettu jopa 12 tunnin yhtäjaksoinen lento dronella.
