\chapter{Anturit}
\label{ch:anturit}

Tässä luvussa kerrotaan dronejen sensoreiden (antureiden)
suunnitteluvaatimuksista, toiminnasta ja tarkoituksesta.  Tekstissä antureita
käsitellään kahdessa osassa, joiden jako perustuu antureiden tarkoitukseen:
\begin{itemize}
\item UAV:n asennon, sijainnin ja tilan kertomiseen tarkoitetut anturit. Tähän
  kastiin kuuluvat esimerkiksi kiihtyvyysanturit, GPS (global positioning
  system) ja lämpötila-anturit.
\item Halutun datan keräykseen tarkoitetut anturit. Esimerkiksi kamerat ja
  etäisyysanturit kuuluvat tähän luokkaan.
\end{itemize}

\section{Dronen tilaa tarkkailevat anturit}
Dronen sijaintia ja asentoa tarkkaillaan yleensä GPS:n ja IMU:n (inertial
measurement unit) avulla. Välillä käytetään myös kompassia.

GPS on Yhdysvaltain puolustusministeriön hallinnoima
satelliittipaikannusjärjestelmä. Satelliittipaikannusjärjestelmät, mukaan
lukien GPS, perustuvat kolmeen tai useampaan satelliittiin, jotka jatkuvasti
lähettävät aikaleiman ja tiedon sijainnistaan suhteessa maahan. Satelliittien
aikaleimojen ja sijantitietojen avulla GPS-vastaanotin pystyy laskemaan oman
sijaintinsa. Oikeassa maailmassa tulee kuitenkin ottaa huomioon muun muassa
myös sääolosuhteet ja maapallon epätasaisuus. Satelliittien kellojen tulee myös
olla äärimmäisen tarkkoja, ja niiden pitää olla synkronoituna keskenään.
Satelliittien etäisyydestä johtuen maapallon painovoima vaikuttaa niihin
vähemmän, jolloin suhteellisuusteorian mukaisesti niiden aika kuluu eri tahtiin
kuin maapallon pinnalla, joka tulee myös huomioida sijainnin
laskennassa. Ja jos vastaanottimen aika on virheellinen, tarvitaan neljäs
satelliitti virheen oikaisemiseksi.~\cite{Parkinson1995}

GPS ei ole ainoa satelliittipaikannusjärjestelmä, mutta se oli ensimmäinen ja
sitä käytetään laajimmin kuluttajasektorilla. Luultavasti näistä syistä GPS on
puhekielessä muodostunut geneeriseksi termiksi
satelliittipaikannusjärjestelmälle. Muita satelliittipaikannusjärjestelmiä ovat
muun muassa EU:n Galileo~\cite{EUGalileo} ja Kiinan BeiDou~\cite{CNBeiDou}.

GPS:n avulla voidaan siis määrittää UAV:n sijainnin maapallolla. IMU:n avulla
saadaan selville mihin suuntaan UAV liikkuu ja mikä sen asento on.  IMU koostuu
yleensä kolmella akselilla toimivasta kiihtyvyysanturista, ja myös kolmella
akselilla toimivasta gyroskoopista. Kiihtyvyysanturit mittaavat
etenemisliikettä ja gyroskoopit kiertoliikettä.~\cite{Mems2017}

Kiihtyvyysanturi koostuu vähintään kolmesta komponentista: Massasta, jonka
suuruus tiedetään tarkasti, massaa pitelevästä jousituksesta, ja mekaanista
liikettä mittaavasta anturista. Dronen kiihtyvyyttä muuttavat voimat
vaikuttavat välillisesti myös kiihtyvyysanturin massaan, jonka kiihtyvyys
aiheuttaa sitä kannattelevaan jouseen jännitteen, jonka anturi voi mitata, ja
joka voidaan taas muuntaa kiihtyvyysarvoksi. Kiihtyvyysarvoja integroimalla
voidaan selvittää myös dronen liikkumisnopeus. Koska anturin mittaama arvo on
kuitenkin skalaari, massan liike täytyy rajoittaa vain yhdelle
akselille.~\cite{Mems2017}

Yleisin tapa toteuttaa gyroskooppi tarpeeksi pienenä, että se mahtuu droneen,
on yllättävän samanlainen kuin kiihtyvyysanturin toteutus. Siinäkin
hyödynnetään testimassaa ja sitä kannattelevaa jousitusta, mutta tällä kertaa
massa värähtelee, ja se voi liikkua kahdella akselilla. Toisella akselilla massa
värähtelee, ja toisella akselilla sen liikettä mitataan kuin
kiihtyvyysanturissakin. Värähtelyn seurauksena gyroskoopin kiertoliike
aiheuttaa massassa Coriolisvoiman, joka liikuttaa massaa mitattavalla
akselilla. Massan liike voidaan mitata ja muuntaa kulmanopeuden
arvoksi.~\cite{Mems2017}

Jokainen kiihtyvyysanturi ja gyroskooppi mittaa kiihtyvyyttä tai kulmanopeutta
vain yhdellä akselilla. Näin ollen kumpiakin tarvitaan kolme, jotta kiihtyvyys
ja kulmanopeus voidaan mitata kolmiulotteisessa avaruudessa.~\cite{Mems2017}

\section{Ulkoiset anturit}
Sotilaskäytössä olevia UAV:ta käytetään usein tarkkailuun. Tällöin niiden
pääasiallinen ulkoinen anturi on kamera, joka voi ottaa kuvia yhdellä tai
useammalla spektrillä. Eri piirteet korostuvat eri spektreillä. Jotkut asiat,
kuten esimerkiksi naamioitunut ihminen, voivat olla näkyvän valon
aallonpituudella lähes näkymättömiä. Ihminen erottuu kuitenkin kehonlämpönsä
takia selvästi infrapunavalon aallonpituudella.~\cite{Modest2013}

Lähteessä~\cite{Hogan2017} on tutkittu dronen
sovelluksia maanviljelyssä. Normaalilla kameralla otettuja kuvia voidaan
hyödyntää esimerkiksi topografisessa mallintamisessa. Infrapunakuvista nähdään
kuvassa olevien kasvien tai eläinten lämpötila, joka voi olla hyödyllistä
karjan tai sadon tilan tarkkailussa.

Tutkimuskäytössä olevat dronet käyttävät myös usein kameraa pääasiallisena
anturinaan. Tällöin ne esimerkiksi voivat ottaa kuvia, joilla voidaan tarkkailla
esimerkiksi metsien tilaa tai kartoittaa maastoa. Etäisyysanturilla,
esimerkiksi LIDAR:lla (light detection and ranging), voidaan kartoittaa maaston
muotoja~\cite{Hogan2017}. Metereologisten ilmiöiden tutkimisessa voi olla
hyödyllistä kerätä lämpötila- ja tuulennopeustietoa.

Droneen on painon, mittojen ja virrankäytön rajoissa mahdollista liittää mitä
vain antureita. Harva UAV on kuitenkaan tarkoitettu laskeutumaan datan keruuta
varten, johtuen laskeutumisen suhteellisen suuresta riskistä epäonnistua.
Laskeutuneena UAV on myös normaalia alttiimpi muun muassa eläinten
hyökkäyksille. Näin ollen droneissa käytetyt anturit ovat pääasiassa sellaisia,
jotka pystyvät mittaamaan pitkiltäkin etäisyyksiltä.
