\nonstopmode{}
% \documentclass[12pt,a4paper,english
% % ,twoside,openright
% ]{tutthesis}
\documentclass[12pt,a4paper,finnish]{tutthesis}

% Note that you must choose either Finnish or English here and there in this
% file.
% Other options for document class
  % ,twoside,openright   % If printing on both sides (>80 pages)
  % ,twocolumn           % Can be used in lab reports, not in theses

% Ensure the correct Pdf size (not needed in all environments)
\special{papersize=210mm,297mm}


% LaTeX file for BSC/MSc theses and lab reports.
% Requires the class file (=template) tutthesis.cls and figure files,
% either tut-logo, exampleFig (as pdf or eps) and example_code.c
% Author: Sami Paavilainen (2006)
% Modified: Heikki Huttunen (heikki.huttunen@tut.fi) 31.7.2012.
%           Erno Salminen, @tut.fi, 2014-08-15
%             - added text snippets from the writing guide
%             - added lots of comments: both tips and alternative styles
%             - added an example table
%             - and so on...



% More information about Latex basics:
% [Tobias Oetiker, Hubert Partl, Irene Hyna, Elisabeth Schlegl, The
% Not So Short Introduction to LATEX2e, Version 5.03, April 2014, 171
% pages.  Availbale: http://tobi.oetiker.ch/lshort/lshort.pdf]


%
% Define your basic information
%
\author{Jonas~Nikula}
\title{Drone-elektroniikka} % primary title (for front page)
% \titleB{Otsikko}     % translated title for abstract
\thesistype{Kandidaatintyö} % or Bachelor of Science, Laboratory Report... 
\examiner{Erja~Sipilä} % without title Prof., Dr., MSc or such

% Put your thesis' main language last
% http://mirrors.ctan.org/macros/latex/required/babel/base/babel.pdf
\usepackage[finnish]{babel}

\graphicspath{{../kuvat/}}
\usepackage[style=ieee]{biblatex}
\bibliography{ele-kandi}


%
% You can include special packages or define new commands here at the
% beginning. Options are given in brackets and package name is in
% braces:  \usepackage{opt]{pkg_name}

% Option1) for bibliography does not need additional packages.

% Option2b) for bibliography: old way for using Name-year citations
% http://www.ctan.org/tex-archive/macros/latex/contrib/harvard/ 
%\usepackage{harvard}  


% Option3) for bibliography: newer way, esp. for Name-year citations
% http://www.ctan.org/pkg/biblatex
%\usepackage[style=authoryear,maxcitenames=2,backend=bibtex,
%  firstinits=true]{biblatex}
%% Note that option style=numeric works as well
%\addbibresource{thesis_refs.bib}


% Don't hyphenate
\hyphenation{Erja}
\hyphenation{Sipilä}


% You can also add your own commands
\newcommand\TODO[1]{{\sethlcolor{yellow}\hl{TODO: #1}}} % Remark text in braces appears in red
\newcommand{\angs}{\textsl{\AA}}              % , e.g. slanted symbol for Ångstöm

% Preparatory content ends here



\pagenumbering{roman} % was: {Roman}
\pagestyle{headings}
\begin{document}



% Special trick so that internal macros (denoted with @ in their name)
% can be used outside the cls file (e.g. \@author)
\makeatletter



%
% Create the title page.
% First the logo. Check its language.
\thispagestyle{empty}
\vspace*{-.5cm}\noindent
\includegraphics[width=8cm]{tty_tut_logo.pdf}   % Bilingual logo



% Then lay out the author, title and type to the center of page.
\vspace{6.8cm}
\maketitle
\vspace{7.7cm} % -> 6.7cm if thesis title needs two lines

% Last some additional info to the bottom-right corner
\begin{flushright}  
  \begin{minipage}[c]{6.8cm}
    \begin{spacing}{1.0}
      \textsf{Tarkastaja:\\*Yliopisto-opettaja \@examiner}\\
      \textsf{Jätetty tarkastettavaksi 13.08.2017}\\ 
      % \textsf{xxxxxxx tiedekuntaneuvoston}\\
      % \textsf{kokouksessa dd.mm.yyyy}\\
      % \textsf{Examiner: Prof. \@examiner}\\
      % \textsf{Examiner and topic approved by the}\\ 
      % \textsf{Faculty Council of the Faculty of}\\
      % \textsf{xxxx}\\
      % \textsf{on 30th July 2014}\\
    \end{spacing}
  \end{minipage}
\end{flushright}

% Leave the backside of title page empty in twoside mode
\if@twoside
\clearpage
\fi



%
% Use Roman numbering I,II,III... for the first pages (abstract, TOC,
% termlist etc)
\pagenumbering{Roman} 
\setcounter{page}{0} % Start numbering from zero because command 'chapter*' does page break





% Foreign students do not need Fininsh abstract (tiivistelmä). Move
% this before English abstract if thesis is in Finnish. Move also the
% otherlanguage command to the English abstract (if needed).
\chapter*{Tiivistelmä} % Asterisk * turns numbering off

\begin{spacing}{1.0}
         {\bf \textsf{\MakeUppercase{\@author}}}: \@title\\  % or use \@title when thesis is in Finnish
         \textsf{Tampereen teknillinen yliopisto}\\
         \textsf{Kandidaatintyö, 35 sivua}\\ %
         \textsf{Elokuu 2017}\\
         \textsf{TST koulutusohjelma}\\
         \textsf{Pääaine: Elektroniikka}\\
         \textsf{Tarkastajat:  Yliopisto-opettaja \@examiner}\\ % automated, if just 1 examiner
         \textsf{Avainsanat: drone, UAV, sulautetut järjestelmät}\\
\end{spacing}


% \hl{
% The abstract in Finnish. Foreign students do not need this page.

% Suomenkieliseen diplomityöhön kirjoitetaan tiivistelmä sekä suomeksi
% että englanniksi.

% Kandidaatintyön tiivistelmä kirjoitetaan ainoastaan kerran, samalla
% kielellä kuin työ. Kuitenkin myös suomenkielisillä kandidaatintöillä
% pitää olla englanninkielinen otsikko arkistointia varten.
% }

Dronet, tai UAV:t (unmanned aerial vehicle), suomeksi miehittämättömät
ilma-alukset, ovat kehittyvä ja kasvava tekniikan alue. Niissä käytettävän
teknologian --- varsinkin tekoälyn --- kehittyessä myös niiden mahdolliset
sovellukset lisääntyvät. Vieläkin niiden merkittävimmät sovelluskohteet ovat
sotilastoiminnassa, mutta onneksi niiden käyttö tutkimuksessa ja
humanitäärisissä tehtävissä on lisääntymässä. Jälkimmäisissä sovelluksissa
dronejen hinnan lasku ja niiden kehittyvä itsenäisyys ovat isoja tekijöitä.

Droneja on myös monenlaisia ja monentasoisia. Harrastekäytössä oleva drone voi
mahtua kämmenelle, pystyy lentämään 15 minuuttia, ja on maksanut alle 50 euroa.
Sotilaskäytössä oleva drone voi olla suurempi kuin lentokone, lentää
vuorokausien ajan monen sadan kilometrin toimintasäteellä, ja on maksanut
vähintään yli miljoona dollaria. Luonnollisesti näiden kahden dronen
vaatimukset ja suunnittelu myös eroavat toisistaan huomattavasti.

Tämän työn tarkoitus on tarkastella dronejen toimintaa mahdollistavaa
teknologiaa, erityisesti niissä olevaa elektroniikkaa. Työssä käsitellään
dronen teholähteitä, sulautettuja järjestelmiä, ohjelmistoa, tekoälyä,
antureita, moottoreita ja kommunikaatiojärjestelmiä. Työssä tarkastellaan myös
tulevaisuudennäkymiä.



% \begin{otherlanguage}{finnish} %  Following text in in 2nd language
% % Some fields in abstract are automated, namely those with \@ (author,
% % title in the main language, thesis type, examiner).
% \chapter*{Abstract}

% \begin{spacing}{1.0}
%          {\bf \textsf{\MakeUppercase{\@author}}}: \@titleB\\   % use \@titleB when thesis is in Finnish
%          \textsf{Tampere University of Technology}\\
%          \textsf{\@thesistype, xx pages, x Appendix pages} \\
%          \textsf{xxxxxx 201x}\\
%          \textsf{Master's Degree Programme in xxx Technology}\\
%          \textsf{Major: }\\
%          \textsf{Examiner: Prof. \@examiner}\\ % 
%          \textsf{Keywords: }\\
% \end{spacing}


% \hl{
% The abstract is a concise 1-page description of the work: what was the
% problem, what was done, and what are the results. Do not include
% charts or tables in the abstract.

% Put the abstract in the primary language of your thesis first and then
% the translation (when that is needed).
% }
% \end{otherlanguage} % End on 2nd language part




\chapter*{Alkusanat}

Haluan kiittää yliopisto-opettaja Erja Sipilää työn tarkastuksesta ja
ohjauksesta. Perhettäni ja ystäväpiiriäni kiitän henkisestä tuesta.
\vspace{25mm}
\newline
Tampereella 13.08.2017
\newline
Jonas Nikula


% This document template conforms to Guide to Writing a Thesis at
% Tampere University of Technology (2014) and is based on the previous
% template. The main purpose is to show how the theses are formatted
% using LaTeX (or \LaTeX ~ to be extra fancy) .

% The thesis text is written into file \texttt{d\_tyo.tex}, whereas
% \texttt{tutthesis.cls} contains the formatting instructions. Both
% files include lots of comments (start with \%) that should help in
% using LaTeX. TUT specific formatting is done by additional settings on
% top of the original \texttt{report.cls} class file. This example needs
% few additional files: TUT logo, example figure, example code, as well
% as example bibliography and its formatting (\texttt{.bst}) An example
% makefile is provided for those preferring command line. You are
% encouraged to comment your work and to keep the length of lines
% moderate, e.g. <80 characters. In Emacs, you can use \texttt{Alt-Q} to
% break long lines in a paragraph and \texttt{Tab} to indent commands
% (e.g. inside figure and table environments). Moreover, tex files are
% well suited for versioning systems, such as Subversion or Git.  
% % \url{http://www.ctan.org/tex-archive/info/lshort/english/lshort.pdf}


% Acknowledgements to those who contributed to the thesis are generally
% presented in the preface. It is not appropriate to criticize anyone in
% the preface, even though the preface will not affect your grade. The
% preface must fit on one page. Add the date, after which you have not
% made any revisions to the text, at the end of the preface.

% ~ 
% % Tilde ~ makes an non-breakable spce in LaTeX. Here it is used to get
% % two consecutive paragraph breaks

% Tampere, 11.8.2014

% ~


% On behalf of the working group, Erno Salminen





% Add the table of contents, optioanlly also the lists of figures,
% tables and codes.

\renewcommand\contentsname{Sisällys} % Set Finnish name, remove this if using English
\setcounter{tocdepth}{3}              % How many header level are included
\tableofcontents                      % Create TOC

% \renewcommand\listfigurename{Kuvaluettelo}  % Set Finnish name, remove this if using English
% \listoffigures                               % Optional: create the list of figures
% \markboth{}{}                                % no headers


% \renewcommand\listtablename{Taulukkoluettelo} % Set Finnish name, remove this if using English
% \listoftables                                  % Optional: create the list of tables
% \markboth{}{}                                  % no headers




%\renewcommand\lstlistlistingname{Ohjelmaluettelo} % SetFinnish name, remove this if using English
%\lstlistoflistings                                % Optional: create the list of program codes
%\markboth{}{}                                     % no headers




%
% Term and symbol exaplanations use a special list type
%

% \chapter*{List of abbreviations and symbols}
\markboth{}{}                                % no headers
% \chapter*{Lyhenteet ja merkinnät}

% You don't have to align these with whitespaces, but it makes the
% .tex file more readable
% \begin{termlist}
% \item [UAV]     Unmanned Aerial Vehicle
% \item [Drone]   UAV:n puhekielinen nimitys
% \item [PEM]     Proton Exchange Membrane
% \item [PEMfc]   Polymer Electrolyte Membrane
% \end{termlist} 
\chapter*{Lyhenteet ja merkinnät}

\begin{termlist}
\setlength\itemsep{1em}

\item [BLDC] englanniksi brushless direct current, on sähkömööttöri, jossa ei
  ole harjoja, jotka muuttavat käämin läpi virtaavan virran suuntaa

\item [Drone] UAV:n puhekielinen nimitys

\item [DSP] englanniksi digital signal processor, on digitaaliseen
  signaalinkäsittelyyn räätälöity mikroprosessori

\item [ESC] englanniksi electronic speed controller, ohjaa sähkömoottorin
  pyörimisnopeutta

\item [GPS] englanniksi global positioning system, Yhdysvaltain
  puolustusministeriön ylläpitämä ja laajalle levinnyt
  satelliittinavigointijärjestelmä

\item [IMU] englanniksi inertial motion unit, siru, joka mittaa asentoa ja
  liikettä

\item [LIDAR] englanniksi light detection and ranging, etäisyysanturi, joka
  toimii lähettämällä kohteeseen laserpulsseja ja mittaamalla niiden
  heijastuksen

\item [MPPT] englanniksi maximum power point tracking, aurinkokennoissa
  käytetty teknologia, jossa kennon jännitettä tai virtaa säädellään, jotta
  kennon tehontuotanto saadaan maksimoitua

\item [Multiroottori] useamman roottorin avulla lentävä UAV

\item [PEM] englanniksi proton exchange membrane, protoneja johtava kalvo, jota
  käytetään polttokennoissa

\item [POMDP] englanniksi partially observable Markov decision process,
  suomeksi osittain havaittavissa oleva Markov-päätösprosessi, on tapa
  mallintaa päätöksentekoa tilanteissa, joissa lopputulos ei ole täysin
  hallittavissa

\item [PWM] englanniksi pulse width modulation, suomeksi
  pulssinleveysmodulaatio, on modulaatiometodi, joka tuottaa kahden jännitteen
  välillä vaihtelevan signaalin

\item [ROS] RobotOperatingSystem, robottien ohjelmistokehitykseen tarkoitettu
  kehitysalusta

\item [SMA] englanniksi shape memory alloy, suomeksi muistimetalli. ``Muistaa''
  aikasemman muotonsa, ja se voidaan palauttaa tähän muistettuun muotoon,
  jollain tavalla, esimerkiksi lämmöllä tai sähkövirralla

\item [UAV] englanniksi unmanned aerial vehicle, miehittämätön ilma-alus

\end{termlist} 
  % Erillinen termlist tiedosto. Toimii includella, muttei inputilla


% The actual text begins here and page numbering changes to 1,2...
% Leave the backside of title empty in twoside mode
\if@twoside
%\newpage
\cleardoublepage
\fi

\pagenumbering{arabic}
\setcounter{page}{1} % Start numbering from zero because command
                     % 'chapter*' does page break
\renewcommand{\chaptername}{} % This disables the prefix 'Chapter' or
                              % 'Luku' in page headers (in 'twoside'
                              % mode)

\chapter{Johdanto}
\label{ch:johdanto}
Miehittämättömät ilma-alukset, eli englanniksi Unmanned Aerial Vehicle:t (UAV)
tai tuttavallisemmin ``dronet'', ovat kokoajan lisääntyvä tekniikan alue.
Asevoimat käyttävät droneja laajasti, joko tiedustelussa tai
taistelukentillä, mutta niille löytyy lisääntyvässä määrin käyttöä myös
siviilimaailmassa. Esimerkiksi Amazon suunnittelee käyttävänsä niitä tavaran
nopeassa ja halvassa kuljetuksessa, ja video- ja valokuvauksessa ne ovat
mahdollistaneet halvat ilmakuvat. Myös tutkimuksessa ja muussa datan
keräyksessä UAV:t ja muut miehittämättömät ajoneuvot tulevat helpottamaan
ihmisten työtä, ja avaruustutkimuksessa miehittämättömät alukset ovat jo
pitkään tehneet työtä, joka on ollut liian kallista tai vaarallista ihmisille.
UAV:den laajoihin käyttöalueisiin kuuluvat myös telekommunikaatioalan
sovellukset, joissa ne ovat halvempia ja joustavampia kuin satelliitit.

Kuvissa~\ref{fig:small_12dollar_quad} ja~\ref{fig:big_predator} näkyy hyvin,
kuinka erilaisia dronet voivat olla; Toinen on 12 dollarin arvoinen,
leikkikäyttöön tarkoitettu, kädenkokoinen malli, ja toinen on päälle 4:n
miljoonan dollarin arvoinen sotilastiedustelumalli, jonka siipiväli on 14,8 m.

\begin{figure}[H]
\centering
\begin{minipage}{.5\textwidth}
  \centering
  \includegraphics[width=0.8\textwidth]{small_12dollar_quad}
  \caption{Leikkikäyttöön tarkoitettu drone~\cite{miniQuad}}
\label{fig:small_12dollar_quad}
\end{minipage}%
\begin{minipage}{.5\textwidth}
  \centering
  \includegraphics[width=0.8\textwidth]{big_predator}
  \caption{Sotilastiedustelumalli, General Atomics MQ-1
    Predator~\cite{bigPredator}}
\label{fig:big_predator}
\end{minipage}
\end{figure}

UAV:t koostuvat monista eri osista. Korkealla tasolla UAV voidaan esimerkiksi
jakaa seuraaviin osiin:
\begin{itemize}

\item teholähde, yleensä akku, mutta voi kehittyneissä malleissa olla myös
esimerkiksi aurinkopaneeli.

\item sulautetut järjestelmät. UAV:n ohjaus, anturidatan keräys ja käsittely,
ja mahdollisesti myös tekoäly pohjautuvat nykyaikana halpaan ja helposti
saataavaan laskentatehoon.

\item anturit. UAV:den pääkäyttö on nykyaikana kaikenlaisen datan kerääminen,
joten anturit ovat melkein kaikkissa droneissa olennaisia.

\item aktuaattorit. Kaikki dronet tarvitsevat ainakin ohjausaktuaattoreita.
Multiroottoreissa (useamman roottorin avulla lentävä drone) tämä tarkoittaa
ainoastaan moottoreiden nopeuden säätöä, mutta lentokonemalleissa tarvitaan myös
siivekkeiden ohjausta.

\item kommunikaatiojärjestelmät. Harrastemalleissa on yleensä yksinkertaisia,
lyhytkantoisia radiojärjestelmiä, joiden tarkoitus on yleensä vain ohjata
dronea. Sotilasmalleissa taas on tarpeen välittää paljon dataa kumpaankin
suuntaan, myös silloin, kun ohjaaja ja alus ovat eri mantereilla.
\end{itemize}

Kuvassa~\ref{fig:harrastedrone} on kuvattu yksinkertaisen harrastemallin
elektroniikkaa ja lohkojen välisiä kytkentöjä. Kyseessä on
multiroottorimalli, joka on harrastajien keskuudessa yleinen sen
yksinkertaisuudesta johtuen. Kuvasta nähdään, että kyseisessä mallissa on
aktuaattoreina neljä moottoria ja neljä nopeudensäädintä. Dronesta löytyy
radiovastaanotin ja lentämistä helpottava lentotietokone. Teholähteenä toimii
yksi litiumioniakku.
\begin{figure}[H]
  \begin{center}
    \includegraphics[width=1.0\textwidth]{harrastedrone}
  \end{center}
  \caption{Harrastedronen elektroniset järjestelmät, perustuu kuvaan lähteessä~\cite{RedditQuad}}
\label{fig:harrastedrone}
\end{figure}


Tämän työn tarkoituksena on käsitellä edellä listattuja UAV:n elektronisia osia
tarkemmin.
Luvussa 2 esitellään, mitä eri teholähteitä UAV:t voivat käyttää.
Luvussa 3 käsitellään dronen sulautettuja järjestelmiä, kuten
suunnistusjärjestelmiä tai automaattiohjausta.
Luvussa 4 käsitellään droneissa käytettyjä antureita ja luvussa 5 niissä
käytettyjä aktuaattoreita.
Luvussa 6 käydään läpi UAV:n ohjauksessa ja tiedonsiirrossa käytettäviä
kommunikaatiojärjestelmiä.
Luvussa 7 on yhteenveto työn sisällöstä.

\hyphenation{tuu-li-ener-gia}

\chapter{Teholähde}
\label{ch:teholahde}

Tässä luvussa käsitellään UAV:den käyttämiä teholähteitä, ja niihin kohdistuvia
erityisvaatimuksia. Useimmissa UAV:ssa moottorit käyvät koko ajan, joka luo
jatkuvan energian tarpeen. Tämän lisäksi anturit ja aktuaattorit voivat myös
luoda piikkejä energian tarpeeseen.

Perinteisesti UAV:t ovat käyttäneet teholähteenään joko akkuja tai
polttomoottoreita. Uudet teholähteet tekevät kumminkin tuloaan.  Esimerkiksi
aurinkovoima soveltuu hyvin varsinkin korkealla toimiviin droneihin, koska
pilvet eivät vaikuta aurinkokennojen tehokkuuteen. Aurinkoenergian lisäksi
ympäristöä voidaan hyödyntää energiasieppausmetodein, esimerkiksi muuntamalla
UAV:n potkurista syntyvää värinää sähköenergiaksi, tai hyödyntämällä
toiminta-alueen ilmavirtauksia liikkumiseen. Myös polttokennot ovat
mahdollinen teholähde, vaikkakin nykyisellään niiden matala energiatiheys
rajoittaa niiden käyttöä lennokeissa~\cite{Bradley2007}.

Hybriditeholähteet ovat lupaavimpia tulevaisuuden teholähteitä, koska
polttomoottoria lukuunottamatta tässä tekstissä käsitellyt teholähteet eivät
kykene tuottamaan tehoa tarpeeksi paljon, tasaisesti tai ennustettavasti, tai
jos pystyvät, ne rajoittavat UAV:n toimintaa liikaa. Polttomoottorin ongelmana
taas on sen aiheuttama saaste, ja joissakin tilanteissa sen aiheuttama melu tai
lämpö voi olla epäedullista, vaikkapa jos halutaan estää dronen havaitseminen.
``Hybridi'' tarkoittaa, että teholähde koostuu useammasta teholähteestä, joka
tarkoittaa yleensä akkua, ja jotain, joka lataa sitä, esimerkiksi
aurinkokennoa.  Jotkut hybriditeholähteet koostuvat jopa kolmesta eri
teholähteestä: Akusta, polttokennosta ja aurinkokennosta.~\cite{Chen2010}

\section{Akut ja polttomoottorit}
Käytännössä kaikki dronet tarvitsevat akun riippumatta siitä, onko se ainoa
teholähde vai ei. Monet tutkimuksen kohteena olevista teholähteistä tuottavat
tehoa melko katkonaisesti. Osa, esimerkiksi energiansieppaus potkurin
värinästä, ei edes pysty huolehtimaan tehontuotannosta kokonaan, vaan niiden
tehtävä on pitkittää toiminta-aikaa. Myös polttomoottoria teholähteenään
käyttävät dronet hyötyvät akusta, koska se mahdollistaa elektroniikan
toiminnan vaikka moottori sammuisikin. Akun avulla sammunut moottori voitaisiin
myös pyrkiä käynnistämään uudelleen ilman ulkopuolista apua.

Litiumioniakkuja on perinteisesti käytetty kulutuselektroniikassa, ja myös
harrastedronet kuuluvat tähän ryhmään. Litiumioniakut ovat kuitenkin
yleistyneet myös isommissa sähköajoneuvoissa, koska niillä on monia etuja
perinteisiin lyijyakkuihin verrattuna. Niiden energiatiheys on korkeampi,
ne ovat pitkäkestoisia, niiden itsepurkautuvuus on vähäistä, ja niillä ei ole
muisti-ilmiötä. Akkuja, joissa esiintyy muisti-ilmiötä, täytyy purkaa tyhjäksi
säännöllisesti, tai sen kapasiteetti alenee. Litiumioniakut ovat kuitenkin
muita akkutyyppejä kalliimpia johtuen litiumin
harvinaisuudesta.~\cite{Sparacino2012}

Polttomoottoreita käytetään yleensä isommissa UAV:ssa, kuten sotilasmalleissa.
Esimerkiksi Yhdysvaltojen ilmavoimien käyttämä MQ-1B Predator käyttää Rotax
914F polttomoottoria~\cite{usaf-predator}, joka tuottaa parhaimmillaan
84,5 kW tehoa~\cite{rotax-914}. Ilmavoimien
nettisivujen~\cite{usaf-predator} mukaan Predatorin kantama on 770 mailia, eli
noin 1239 kilometriä.

\sloppy{Suurin syy, miksi isommat UAV:t käyttävät polttomoottoreita akkujen
  sijaan, on polttoaineiden huomattavasti suurempi energiatiheys. Esimerkiksi
  Panasonicin NCR18650B~\cite{panasonic-ncr18650b} litiumioniakun energiatiheys
  on noin 0.87 MJ/kg, kun taas yleisesti käytössä olevan lentokonepolttoaineen
  AVGAS 100LL~\cite{shell-avgas}, jota myös Predator käyttää, energiatiheys on
  noin 44 MJ/kg, joka on 50 kertainen verrattuna litiumioniakkuun.
}



\section{Polttokennot teholähteenä}
Polttokenno on laite joka muuntaa kemiallista energiaa sähköenergiaksi.
Kemiallinen energia syntyy kennoon syötetystä vetypitoisesta polttoaineesta
ja hapesta, tai muusta oksidoivasta aineesta, joiden välinen sähkökemiallinen
reaktio vapauttaa energiaa.

Droneissa polttokennoja käytetään yleensä hybriditeholähteissä, joissa
on sekä polttokenno että akku. Näin teholähde pystyy vastaamaan esimerkiksi
nousun vaatimiin tehopiikkeihin. Kun UAV on lentokorkeudella ja tehonkulutus
matalaa ja tasaista, polttokenno voi ladata akkua. Haittana
hybriditeholähteessä on teholähteen monimutkaisempi rakenne.~\cite{Gao2005}

Lähteessä~\cite{Bradley2007} käsitellään polttokennoa käyttävän UAV-prototyypin
ominaisuuksia ja suorituskykyä. Kyseinen protyyppi käyttää ainoastaan
polttokennoa tehonlähteenään. Polttokenno huipputeho on 465 W ja sen
energiatiheys on noin 0.187 MJ/kg. Prototyyppia on testattu noin 110 s
lennoilla, joiden aikana se on lentänyt noin 1200~m matkan.  Näiden testien
perusteella UAV:n lentoajaksi arvioidaan maksimissaan noin 43 minuuttia.

\section{Aurinkokennot ja energiansieppaus}
Aurinkokennoja ja energiansieppausta voidaan myös hyödyntää
hybriditeholähteissä pidentämään UAV:den lentoaikoja. Energiansieppaus
tarkoittaa ympäristössä ilmenevän energian muuttamista käytettävään
sähköenergiaan~\cite{Anton2011}. Aurinko- ja tuulienergia ovat jo melko
jokapäiväisiä energiansieppausmenetelmiä, mutta esimerkiksi myös ihmisen
ruumiinlämpöä tai moottorin värinää voidaan muuntaa sähköenergiaksi.

Aurinkoenergian ja muiden UAV:n ulkopuolisiin ilmiöihin perustuvien
energiansieppaustapojen tehontuotantoa on kuitenkin vaikea varmistaa, koska
niihin ei pysty vaikuttamaan juuri muuten kuin UAV:n toiminta-alueen
valinnalla. Akulla toimintavarmuutta voidaan parantaa vähentämällä UAV:n
riippuvuutta ulkoisista ilmiöistä.  Niiden järkevät käyttöalueet ja -tilanteet
ovat joka tapauksessa rajattuja; Päiväntasaajalla aurinkoenergia on
varteenotettava vaihtoehto, Suomessa ei niinkään.  Parhaassakin tapauksessa
tehonkulutusta pitää suunnitella tarkaan.  Lähteessä~\cite{Jaw-KuenShiau2009}
käsitellyn prototyypin 3:n aurinkopaneelin yhteen laskettu huipputeho on noin
72 W. Aurinkokennojen tuottama teho riippuu kuitenkin niiden jännitteestä,
joten ne käytännössä vaativat maksimitehopisteen seurantajärjestelmän (Maximum
Power Point Tracking, MPPT). Myös akun lataaminen vaatii lisää elektroniikkaa,
koska aurinkokennojen jännitetaso on yleensä liian matala, jolloin tarvitaan
hakkureita nostamaan jännite käytettävälle tasolle.

Energiaa voidaan siepata esimerkiksi dronen omasta värinästä~\cite{Priya2009}
tai ympäröivistä sääilmiöistä. Värinä on siepattavan energian lähteenä melko
luotettava verrattuna sääilmiöihin, jotka riippuvat muun muassa sijainnista,
vuoden- ja vuorokaudenajasta sekä lentokorkeudesta. Hyvä esimerkki sääilmiön
hyödyntämisestä on ilmavirtausten käyttäminen UAV:n korkeuden lisäämiseen
moottorin sijaan~\cite{Cutler2010}.

\section{Langaton tehonsiirto}
Langattomalla tehonsiirrolla voidaan poistaa tarve laskeutua akun latausta
varten, mikä helpottaa latausta varsinkin malleissa, jotka tarvitsevat
kiitoradan. Latausasemia voidaan sijoittaa maastoon joihin laskeutuminen ei
olisi mahdollista, jolloin UAV pystyy toimimaan kyseisellä alueella paljon
pidempään.

Lähteessä~\cite{Griffin2012} käsitellään magneettiseen resonanssiin perustuvaa
langatonta tehonsiirtoa. Kyseisessä artikkelissa ideana on siirtää energiaa
UAV:n avulla maassa sijaitsevalle anturille, mutta kyseistä menetelmää
voidaan myös käyttää toisinpäin lataamaan UAV:ta. Suurimpana ongelmana
kyseisessä menetelmässä on UAV:n asennon ja sijainnin pitäminen optimaalisena,
mutta siitä huolimatta menetelmä saavutti lähes 5 W jatkuvan tehonsiirron.

Toinen langattoman tehonsiirron teknologia perustuu laseriin.
Lähteessä~\cite{Achtelik2011} on esitetty järjestelmä, jossa tukiasema tähtää
lasersäteen aurinkokennoon, joka on liitetty droneen. Kyseisellä järjestelmällä
on saavutettu jopa 12 tunnin yhtäjaksoinen lento dronella.

\chapter{Sulautetut järjestelmät}
\label{ch:sulautetut}

Sulautetuksi järjestelmäksi kutsutaan yleensä tietokonejärjestelmiä, jotka on
suunniteltu rajattua tarkoitusta varten, ja jotka ovat osa jotain isompaa
järjestelmää. Nykyään kyseinen määritelmä ei ole täysin yksiselitteinen.
Perinteistä puhelinta voidaan pitää sulautettuna järjestelmänä, mutta nykyiset
älypuhelimet muistuttavat jo enemmän yleiskäyttöisiä tietokoneita.
Sulautetuilla järjestelmillä on kuitenkin muutamia yhteisiä piirteitä, jotka
erottavat ne yleiskäyttöisistä tietokoneista.~\cite{Noergaard2005}
\begin{itemize}
  \item Niiden fyysinen koko ja/tai laskentateho on selkeästi rajallisempi.
  \item Ne on suunniteltu rajattua tarkoitusta varten.
  \item Niillä on reaaliaikaisuus-vaatimuksia tai niiden toiminnan tulee olla
    muuten luotettavampaa.
\end{itemize}

UAV:n tietokonejärjestelmät ovat melko selkeästi sulautettuja järjestelmiä.
Niiden toiminnalla on tietty tarkoitus, niiden koko ja teho on rajattu
tiettyihin vaatimuksiin, ja niiden tulee ehdottomasti toimia reaaliaikaisesti
ja luotettavasti.

Tässä luvussa kerrotaan dronejen sulautetuista järjestelmistä, ja dronen niihin
kohdistamista erityisvaatimuksista. Ensin perehdytään sulautettujen
järjestelmien arkkitehtuuriin, eli järjestelmien elektroniikan ja ohjelmiston
suunnitteluun droneille. Esille otetaan myös muutamia kehitysalustoja ja
kirjastoja, jotka ovat tarkoitettu dronen sulautettujen järjestelmien
rakentamiseen. Sen jälkeen käydään tarkemmin läpi näiltä järjestelmiltä
vaadittavaa toiminnallisuutta.

\section{Elektroniikka}
Pienemmät UAV:t, joita käytetään harrastuksessa tai tutkimuksessa, ovat paljon
yksinkertaisempia. Harrastedroneille on tarjolla todella paljon erilaisia
lennonohjauslevyjä, kuten esimerkiksi PyroDrone~\cite{PyroDroneFC}. Kyseinen
levy on melko tyypillinen harrastedronen lennonohjauslevy. Sen keskiössä on
32-bittinen STM32F405 ARM-arkkitehtuuriin perustuva prosessori, jonka
toiminnallisuutta on jatkettu muun muassa flash-muistilla, monenlaisilla
liitännöillä ja MPU6000 inertiamittauspiirillä.

Linkboard~\cite{Wzorek2017} on pienikokoisille UAV:lle suunniteltu alusta,
jonka laitteisto on muunneltavissa riippuen siitä, kuinka paljon laskentatehoa
tarvitaan. Minimissään laittesto pystyy lennonohjaukseen ja anturidatan
käsittelyyn. Siihen voi kuitenkin lisätä kaksi Gumstix lisälevyä. Lisälevyt
ovat identtisiä, ja sisältävät yhden ARM Cortex-A8 prosessorin ja digitaalisen
signaalinkäsittelypiirin (digital signal processor, DSP). Lisälevyjen avulla
Linkboard pystyy korkeatasoiseen lennon suunnitteluun ja datan käsittelyyn.

\section{Ohjelmisto}
Koska kyseessä on sulautettu järjestelmä, ohjelmiston toteutus riippuu pitkälti
käytetystä elektroniikasta. Yksinkertaisissa malleissa ei välttämättä käytetä
käyttöjärjestelmää, vaan mikrokontrolleri suorittaa yhtä ohjelmaa, joka ohjaa
dronea. Toiminnallisuuden kasvaessa ohjelmisto voi kasvaa hallitsemattomaksi,
jolloin on hyödyllistä käyttää olemassaolevia kehitysalustoja.

Yksi esimerkki tälläisestä kehitysalustasta on Robot Operating System (ROS),
joka kerää yhteen työkaluja ja kirjastoja helpottamaan robottia ja
miehittämättömien ajoneuvojen ohjelmistokehitystä~\cite{RobotOperatingSystem}.
Erityisesti droneille on olemassa myös flyt\@{OS}. Flyt\@{OS} toimittaa
pitkälti samaa virkaa kuin ROS, mutta se keskittyy erityisesti
drone-sovelluksiin ja niiden vaatimuksiin.~\cite{flytOS}

Kauko-ohjattaville harrastelennokeille ja multiroottoreille on olemassa
lukuisia lennonohjausohjelmistoja. Yksi esimerkki on Cleanflight. Cleanflight
on suhteellisen helposti konfiguroitava ohjelmisto, joka muun muassa helpottaa
lennokin ohjausta PID-säädöin ja mahdollistaa vaikkapa LED-nauhojen
kauko-ohjauksen.~\cite{Cleanflight}


\section{Toiminnallisuus}

Kuvassa~\ref{fig:uav_ohjaus_tasot} on periaatekuva UAV:n ohjauksen tasoista.
Tasot on kuvattu laskevan abstraktion mukaan; Ylimpänä on melko käsitteellinen
``Tehtävä'', alimpana todella yksityiskohtaista hallintaa vaativat
aktuaattorit. Katkoviivalla on rajattu tasot, jotka voivat kuulua UAV:n
sulautettujen järjestelmien toiminnallisuuteen.
\begin{figure}[H]
  \begin{center}
    \includegraphics[width=1.0\textwidth]{uav_ohjaus_tasot}
  \end{center}
  \caption{Periaatekuva UAV:n ohjauksen tasoista.}
\label{fig:uav_ohjaus_tasot}
\end{figure}

Kauko-ohjattavan harrastedronen tietokonejärjestelmien toiminnallisuudella ei
ole paljon vaatimuksia. Vaatimuksena on lähinnä helpottaa ohjausta säätämällä
moottoreita.  Mahdollisesti halutaan myös ohjata vaikkapa dronesta löytyviä
LED:ejä.  Itsenäisen, tiedusteluun tai tiedon keräykseen tarkoitetun, UAV:n
ohjelmisto on jo astetta monimutkaisempi. Enää ei riitä vain moottorin säätö,
tarvitaan myös täysin varusteltu autopilotti, ja itsenäisyyden tasosta riippuen
ehkä myös jonkun tason tekoäly, joka päättää mitä UAV seuraavaksi tekee. Jos UAV
lentää matalalla, sen tulee pystyä väistämään maaston piirteitä, ja on myös
mahdollista, että UAV:n halutaan itse pystyvän, jossain määrin, tulkitsemaan
havaintojaan.
Tämän takia tässä tekstissä on eroteltu yksinkertaisempi
autopilotti ja monimutkaisempi tekoäly omiin alakappaleisiinsa.

\subsection{Autopilotti}
Perinteisen autopilotin toiminnallisuus voidaan jakaa kahteen osaan: Lentävän
koneen, esimerkiksi lentokoneen tai dronen, vakautus, ja ohjaus, joka pitää
koneen reitillään. Vakautus pyrkii pitämään koneen mahdollisimman vakaana,
ohjaamalla esimerkiksi jatkuvaa sivutuulta vastaan. Lennonvakausjärjestelmä voi
myös helpottaa ohjausta abstrahoimalla esimerkiksi moottorinnopeuden säädön.
Reitinohjaus taas pyrkii pitämään koneen halutulla reitillä. Modularisuuden
kannalta on hyödyllistä suunnitella vakaus ja ohjaus erillisinä lohkoina, niin
kuin lähteessä~\cite{Capello2017}, jossa on kuvattu autopilotti, jonka
reitinohjausjärjestelmä ohjaa vakautusjärjestelmää.

Harrastedroneissa on yleensä käytössä vain lennonvakausjärjestelmä. Nykyisin
suosittuja multiroottoreita olisi huomattavasti vaikeampi ohjata, jos ohjaaja
joutuisi säätämään suoraan jokaisen moottorin pyörimisnopeutta. Sen sijaan
ohjaaja voi antaa dronelle komennon kallistua vasemmalle. Tämä on mahdollista
lennonohjauslevyllä olevan säädön seurauksena. Harrastedronejen
lennonohjauslevyt käyttävät yleensä PID-säädintä. PID-säädin on suljetun
silmukan säädin, joka tarkoittaa, että säädin huomioi ulostulossaan säätönsä
vaikutuksen. Droneissa tämä tarkoittaa, että se tarkkailee drone korkeuden,
nopeuden ja asennon muutosta asettaessaan moottoreiden
pyörimisnopeuden. Se ottaa antureiden datan huomioon säädössään, jolloin, jos
drone ei kallistu tarpeeksi, se voi lisätä moottoreiden kierroksia, kunnes
kallistuma on tarpeeksi suuri.~\cite{Sebbane2015}

PID-säädin on vanha ja yksinkertainen keksintö, mutta se on edelleen usein
paras ratkaisu käytännön säätöongelmaan. PID-säädin saa nimensä eri
toimintalohkoistaan; P-lohkon ulostulo on halutun arvon ja nykyisen arvon
erotus kerrottuna vakiolla. I-lohko integroi erotusta ajan yli, eli käytännössä
summaa erotuksen arvoja asetetulta ajalta, ja kertoo summan vakiolla. D-lohko
derivoi erotusta, eli laskee, kuinka nopeasti erotus muuttuu, ja kertoo
derivaatan vakiolla. PID-säätimen muutettavia parametreja ovat jokaisen lohkon
kerroin, ja integraalin ja derivaatan aika. Näitä parametreja muuttamalla
säädin voidaan muokata sopivaksi monenlaisiin
säätötehtäviin.~\cite{Visioli2006}

Keskeistä lennonohjauksessa on UAV:n liikeradan mallintaminen ja suunnittelu.
Etenkin kiinteäsiipisen UAV:n on otettava huomioon monimutkaisia
lentodynamiikkoja. Se ei esimerkiksi saa asettaa konetta liian jyrkkään
nousuun, ettei kone sakkaa. Yksinkertaisempi ohjausjärjestelmä saattaa ottaa
huomioon ainoastaan koneen omat ominaisuudet, esimerkiksi sen siipien nosteen
ja moottorien tehon. Monimutkaisemmat järjestelmät voivat ottaa huomioon myös
meteorologista tietoa, esimerkiksi ohittamalla, alueet joilla on kova
vastatuuli.~\cite{Sebbane2015}

\subsection{Tekoäly ja konenäkö}
Jos autopilotin tehtävä on pitää UAV reitillään, UAV:n tekoälyn tulee valita
seurattava reitti. Reitin valintaan vaikuttavat monet seikat, kuten
suoritettava tehtävä, ympäristö, ja käytettävissä olevat anturit ja
aktuaattorit.

Robotiikassa, johon itsenäiset UAV:t lasketaan~\cite{Sebbane2014}, tekoälystä
puhutaan rationaalisena toimijana. Tässä yhteydessä toimija, englanniksi agent,
voi tarkoittaa mitä tahansa objektia, joka havaitsee ympäristöään antureiden
kautta, ja toimii siinä aktuaattoreiden avulla. Rationaalinen tarkoittaa, että
toimija päättää jokaisessa tilanteessa tehdä sen toiminnon, jonka odotettu
seuraus on mahdollisimman hyvä. Toiminnan hyvyyden arviointi on kuitenkin
vaikea ja keskeinen ongelma tekoälyn kehityksessä.~\cite{Russell2013}
Antureita ja aktuaattoreita käsitellään tarkemmin kappaleissa~\ref{ch:anturit}
ja~\ref{ch:aktuaattorit}.

Reitinsuunnitteluun löytyy monia menetelmiä. Käytetty menetelmä riippuu myös,
minkä tasoisesta reitinsuunnittelusta on kyse. Jos on tarve päästä paikasta A
paikkaan B, ja ympäristö tunnetaan tarkasti, on tarjolla useita
reititysalgoritmeja.~\cite{Sebbane2014} Jos taas tilanne on muuttuva,
esimerkiksi, koska tehtävänä on seurata liikkuavaa kohdetta, tarvitaan
monimutkaisempia ratkaisuja. Artikkelissa~\cite{Ragi2013} on käytetty 
osittain havaittavissa olevaa Markov-päätöksentekoprosessia (partially
observable Markov decision process, POMDP).
Artikkelissa tarkoituksena on ohjata useampaa UAV:ta muuttuvassa ympäristössä,
joiden tulee seurata kohdettaan, kuitenkin samalla pitäen etäisyyttä uhkiin.

Reitinsuunnittelua varten UAV:n täytyy myös pystyä mallintamaan ympäristöään.
Suoraan antureista saatu data, kuten kuvat tai LIDAR:sta saatu etäisyysdata,
täytyy käsitellä konenäön avulla.
Konenäkö on tekoälyn osa-alue, jonka tarkoituksena on mahdollistaa ympäristön
havainnointi koneille yhtä tehokkaasti kuin ihmiselle. Konenäössäkin on
useampia tasoja. Alemmalla tasolla käsitellään kuvia, jotta halutut asiat
korostuisivat. Voidaan esimerkiksi korostaa reunoja. Korkeammalla tasolla
käytetään koneoppimisen metodeja, ja koitetaan esimerkiksi tunnistaa, mitä kuva
esittää.~\cite{Davies2012}

Konenäöllä voi olla erilaisia tavoitteita.
Sillä voidaan havainnoida ympäristöä navigointia varten, ja tunnistaa esteitä.
Artikkelissa~\cite{Subramanian2006} on kehitetty järjestelmä sitruspuiden
lomassa ajavan itsenäisen robotin ohjausta varten. Kameralla otettuja kuvia
analysoidaan konenäköjärjestelmällä, joka luo mallin ympäristöstä, jonka avulla
robotti voi ohjata itseään ilman tarkkaa GPS-signaalia.
Konenäköä voi myös käyttää pelastustoimissa ihmisen löytämiseen
maastossa~\cite{Rudol2008}, tai villieläinkantojen
tarkkailuun~\cite{Gonzalez2016}.


\chapter{Anturit}
\label{ch:anturit}

Tässä luvussa kerrotaan dronejen sensoreiden (antureiden)
suunnitteluvaatimuksista, toiminnasta ja tarkoituksesta.  Tekstissä antureita
käsitellään kahdessa osassa, joiden jako perustuu antureiden tarkoitukseen:
\begin{itemize}
\item UAV:n asennon, sijainnin ja tilan kertomiseen tarkoitetut anturit. Tähän
  kastiin kuuluvat esimerkiksi kiihtyvyysanturit, GPS (global positioning
  system) ja lämpötila-anturit.
\item Halutun datan keräykseen tarkoitetut anturit. Esimerkiksi kamerat ja
  etäisyysanturit kuuluvat tähän luokkaan.
\end{itemize}

\section{Dronen tilaa tarkkailevat anturit}
Dronen sijaintia ja asentoa tarkkaillaan yleensä GPS:n ja IMU:n (inertial
measurement unit) avulla. Välillä käytetään myös kompassia.

GPS on Yhdysvaltain puolustusministeriön hallinnoima
satelliittipaikannusjärjestelmä. Satelliittipaikannusjärjestelmät, mukaan
lukien GPS, perustuvat kolmeen tai useampaan satelliittiin, jotka jatkuvasti
lähettävät aikaleiman ja tiedon sijainnistaan suhteessa maahan. Satelliittien
aikaleimojen ja sijantitietojen avulla GPS-vastaanotin pystyy laskemaan oman
sijaintinsa. Oikeassa maailmassa tulee kuitenkin ottaa huomioon muun muassa
myös sääolosuhteet ja maapallon epätasaisuus. Satelliittien kellojen tulee myös
olla äärimmäisen tarkkoja, ja niiden pitää olla synkronoituna keskenään.
Satelliittien etäisyydestä johtuen maapallon painovoima vaikuttaa niihin
vähemmän, jolloin suhteellisuusteorian mukaisesti niiden aika kuluu eri tahtiin
kuin maapallon pinnalla, joka tulee myös huomioida sijainnin
laskennassa. Ja jos vastaanottimen aika on virheellinen, tarvitaan neljäs
satelliitti virheen oikaisemiseksi.~\cite{Parkinson1995}

GPS ei ole ainoa satelliittipaikannusjärjestelmä, mutta se oli ensimmäinen ja
sitä käytetään laajimmin kuluttajasektorilla. Luultavasti näistä syistä GPS on
puhekielessä muodostunut geneeriseksi termiksi
satelliittipaikannusjärjestelmälle. Muita satelliittipaikannusjärjestelmiä ovat
muun muassa EU:n Galileo~\cite{EUGalileo} ja Kiinan BeiDou~\cite{CNBeiDou}.

GPS:n avulla voidaan siis määrittää UAV:n sijainnin maapallolla. IMU:n avulla
saadaan selville mihin suuntaan UAV liikkuu ja mikä sen asento on.  IMU koostuu
yleensä kolmella akselilla toimivasta kiihtyvyysanturista, ja myös kolmella
akselilla toimivasta gyroskoopista. Kiihtyvyysanturit mittaavat
etenemisliikettä ja gyroskoopit kiertoliikettä.~\cite{Mems2017}

Kiihtyvyysanturi koostuu vähintään kolmesta komponentista: Massasta, jonka
suuruus tiedetään tarkasti, massaa pitelevästä jousituksesta, ja mekaanista
liikettä mittaavasta anturista. Dronen kiihtyvyyttä muuttavat voimat
vaikuttavat välillisesti myös kiihtyvyysanturin massaan, jonka kiihtyvyys
aiheuttaa sitä kannattelevaan jouseen jännitteen, jonka anturi voi mitata, ja
joka voidaan taas muuntaa kiihtyvyysarvoksi. Kiihtyvyysarvoja integroimalla
voidaan selvittää myös dronen liikkumisnopeus. Koska anturin mittaama arvo on
kuitenkin skalaari, massan liike täytyy rajoittaa vain yhdelle
akselille.~\cite{Mems2017}

Yleisin tapa toteuttaa gyroskooppi tarpeeksi pienenä, että se mahtuu droneen,
on yllättävän samanlainen kuin kiihtyvyysanturin toteutus. Siinäkin
hyödynnetään testimassaa ja sitä kannattelevaa jousitusta, mutta tällä kertaa
massa värähtelee, ja se voi liikkua kahdella akselilla. Toisella akselilla massa
värähtelee, ja toisella akselilla sen liikettä mitataan kuin
kiihtyvyysanturissakin. Värähtelyn seurauksena gyroskoopin kiertoliike
aiheuttaa massassa Coriolisvoiman, joka liikuttaa massaa mitattavalla
akselilla. Massan liike voidaan mitata ja muuntaa kulmanopeuden
arvoksi.~\cite{Mems2017}

Jokainen kiihtyvyysanturi ja gyroskooppi mittaa kiihtyvyyttä tai kulmanopeutta
vain yhdellä akselilla. Näin ollen kumpiakin tarvitaan kolme, jotta kiihtyvyys
ja kulmanopeus voidaan mitata kolmiulotteisessa avaruudessa.~\cite{Mems2017}

\section{Ulkoiset anturit}
Sotilaskäytössä olevia UAV:ta käytetään usein tarkkailuun. Tällöin niiden
pääasiallinen ulkoinen anturi on kamera, joka voi ottaa kuvia yhdellä tai
useammalla spektrillä. Eri piirteet korostuvat eri spektreillä. Jotkut asiat,
kuten esimerkiksi naamioitunut ihminen, voivat olla näkyvän valon
aallonpituudella lähes näkymättömiä. Ihminen erottuu kuitenkin kehonlämpönsä
takia selvästi infrapunavalon aallonpituudella.~\cite{Modest2013}

Lähteessä~\cite{Hogan2017} on tutkittu dronen
sovelluksia maanviljelyssä. Normaalilla kameralla otettuja kuvia voidaan
hyödyntää esimerkiksi topografisessa mallintamisessa. Infrapunakuvista nähdään
kuvassa olevien kasvien tai eläinten lämpötila, joka voi olla hyödyllistä
karjan tai sadon tilan tarkkailussa.

Tutkimuskäytössä olevat dronet käyttävät myös usein kameraa pääasiallisena
anturinaan. Tällöin ne esimerkiksi voivat ottaa kuvia, joilla voidaan tarkkailla
esimerkiksi metsien tilaa tai kartoittaa maastoa. Etäisyysanturilla,
esimerkiksi LIDAR:lla (light detection and ranging), voidaan kartoittaa maaston
muotoja~\cite{Hogan2017}. Metereologisten ilmiöiden tutkimisessa voi olla
hyödyllistä kerätä lämpötila- ja tuulennopeustietoa.

Droneen on painon, mittojen ja virrankäytön rajoissa mahdollista liittää mitä
vain antureita. Harva UAV on kuitenkaan tarkoitettu laskeutumaan datan keruuta
varten, johtuen laskeutumisen suhteellisen suuresta riskistä epäonnistua.
Laskeutuneena UAV on myös normaalia alttiimpi muun muassa eläinten
hyökkäyksille. Näin ollen droneissa käytetyt anturit ovat pääasiassa sellaisia,
jotka pystyvät mittaamaan pitkiltäkin etäisyyksiltä.

\chapter{Aktuaattorit}
\label{ch:aktuaattorit}

Tässä luvussa käsitellään UAV:n liikkumiseen tarvittavia aktuaattoreita, eli
pääasiassa sähkömoottoreita. Droneissa voi olla muitakin aktuaattoreita.
Sotilasmallit saattavat kantaa aseita, joiden laukaisuun tarvitaan
aktuaattoreita. Tutkimus- tai kuljetuskäyttöön tarkoitetuissa malleissa voi
olla tavaroiden kuljetukseen, tai tutkimuslaitteiden keräämiseen ja
purkamiseen, tarkoitettuja aktuaattoreita. Viihdemalleissa voi olla valoja tai
kaiuttimia.

Näitä aktuaattoreita on kuitenkin niin monenlaisia, että niiden käsittely ei
ole järkevää. Kaikissa droneissa on kuitenkin liikumiseen tarkoitettuja
aktuaattoreita, ja monet muut aktuaattorit, kuten kourat ja asejärjestelmät,
käyttävät myös sähkömoottoreita, joita tässä käsitellään.

Ensin tuodaan esille erilaisia dronekokoonpanoja, jonka jälkeen perehdytään
niiden käyttämiin ohjausaktuaattoreihin ja niiden eroihin.

\section{Dronekokoonpanot}

Kuten miehitetyt lentokoneet, myös UAV:t ja dronet voidaan jakaa
kiinteäsiipisiin ja roottorilentokoneisiin, riippuen siitä, millä tavalla ne
pysyvät ilmassa. Kiinteäsiipisten mallien tulee liikkua tarpeeksi nopeasti,
jotta niiden siivet pystyvät luomaan nostetta. Roottorilentokoneet taas pysyvät
ilmassa vain propellien luoman nosteen avulla. Roottorilentokoneiden reititys
on yksinkertaisempaa, ja ne ovat joustavampia, koska niiden ei tarvitse pysyä
liikkeessä pysyäkseen ilmassa. Kiinteäsiipiset mallit taas ovat
energiatehokkaampia, ja pystyvät kuljettamaan painavampia lasteja, koska
moottorin ei tarvitse yksin huolehtia nosteesta.~\cite{Austin2010}

On olemassa myös ratkaisuja, jotka eivät sovi tähän jakoon, kuten esimerkiksi
ornithoptereita~\cite{Austin2010}, mutta suurimmalle osasta droneista kyseinen
jako toimii hyvin.

Roottorilentokoneita on monenlaisia. Pääroottoreita voi olla yksi, kuten
perinteisessä helikopterissa. Tällöin tarvitaan myös toissijainen roottori
vastustamaan pääroottorin luomaa kierremomenttia. Vaihtoehtoisia malleja ovat
koaksiaaliroottori, jossa on kaksi pääroottoria päällekkäin, pyörien
vastakkaisiin suuntiin, näin kumoten toisen aiheuttaman kierremomentin;
Tandemroottori, jossa on kaksi pääroottoria eri kohdissa dronea; Ja
quad-roottori, joka on varsinkin harrastedroneissa suosittu konfiguraatio.
Quad-roottorin etu harrastedroneissa on se, että dronen hallintaan riittää
moottoreiden nopeuden säätö. Muissa konfiguraatiossa täytyy pystyä hallitsemaan
roottorin siivekkeiden kulmaa, joka lisää monimutkaisuutta ja
hintaa.~\cite{Austin2010}

% \TODO{paulos tähän}
% \cite{Paulos2013}
% \cite{Paulos2015}

Kiinteäsiipisissä droneissa moottori on yleensä takana. Näin dronen etuosaan on
mahdollista sijoittaa esimerkiksi antureita, jossa niillä on parempi näkyvyys.
Kiinteäsiipisten konfiguraatioiden isoin ero tulee perän sijainnista. Se voi
olla osa runkoa, kuten useimmissa matkustajalentokoneissa.  Se voi olla myös
kiinni puomeilla, tai se voi puuttua kokonaan, jolloin kyse on lentävästä
siivestä.~\cite{Austin2010}


\section{Sähkömoottorit}
Pienemmissä droneissa käytetään yleensä sähkömoottoreita, koska ne
ovat polttomoottoreita kevyempiä, mutta kuitenkin tarpeeksi
tehokkaita~\cite{Austin2010}. Yksinkertainen, esimerkiksi harrastedronessa
esiintyvä, moottorijärjestelmä koostuu teholähteestä, nopeudensäätimestä,
sähkömoottorista ja propellerista~\cite{Gabriel2011}. Teholähteitä käsiteltiin
jo kappaleessa~\ref{ch:teholahde}, ja propellerin aerodynamiikan käsittely ei
kuulu tämän työn aihepiiriin. Käsiteltäväksi jäävät siis sähkömoottori ja
nopeudensäädin.

Kaikkien sähkömoottoreiden toiminta perustuu sähkövirran aiheuttamaan
magneettikenttään. Kahden magneettikentän kohdatessa syntyy voima, joka pyrkii
asettamaan magneettikentät samansuuntaisiksi. Jos siis on olemassa esimerkiksi
kestomagneetilla luotu magneettikenttä, ja siihen viedään johto, jonka läpi
virtaa sähköä, magneettinen voima pyrkii asettamaan johdon ja kestomagneetin
niin, että niiden magneettikentät ovat samansuuntaiset. Jos tässä tilanteessa
kestomagneetin liike estetään, vain johto muuttaa asentoaan. Jos johdon läpi
kulkevan virran suuntaa vaihdellaan jaksottaisesti, voidaan saada aikaan
pyörimisliike.~\cite{Gottlieb1997}

Kuvassa~\ref{fig:dc-motor} on yksinkertaistettu kuva tasavirtamoottorista.
Kuvassa nähdään kestomagneetin navat, N ja S, käämi, eli johto, jonka läpi
kulkee virta, teholähde ja harjat. Harjat ovat kuvassa teholähteestä lähtevät
kappaleet, jotka osuvat käämiin. Silmukan pyöriessä harjojen kontaktipinta
vaihtuu, jolloin silmukan läpi virtaavan sähkön suunta vaihtuu.  Ilman tätä
virran suunnan vaihtumista silmukka ei pyörisi.~\cite{Gottlieb1997}
\begin{figure}[H]
  \begin{center}
    \includegraphics[width=1.0\textwidth]{dc-motor}
  \end{center}
  \caption{Yksinkertaistettu kuva tasavirtamoottorista.~\cite{Gottlieb1997}}
\label{fig:dc-motor}
\end{figure}

Droneissa käytetyt sähkömoottorit ovat yleensä harjattomia
tasavirtamoottoreita. Harjattomat tasavirtamoottorit, lyhennettynä BLDC
(brushless direct current) moottorit, ovat suosittuja droneissa, koska niiden
ominaisuudet vastaavat hyvin dronen moottorille asettamia
vaatimuksia.~\cite{Gabriel2011}

Näitä ominaisuuksia BLDC-moottoreissa ovat~\cite{Yedamale2003}
\begin{itemize}
  \item Parempi pyörimisnopeus/vääntömomentti suhde. Vääntömomentti ei vähene
    nopeuden kasvaessa.

  \item Parempi hyötysuhde. Jännite ei laske harjojen yli.

  \item Suurempi lähtöteho kokoonsa nähden.

  \item Suurempi nopeusalue. Harjat eivät rajoita nopeutta.

  \item Aiheuttavat vähemmän elektromagneettista häiriötä.

\end{itemize}

Harjojen puute on myös itsessään etu, sillä harjat kuluvat nopeasti, joten
harjattoman moottorin elinikä on pidempi. Harjattomuus johtuu siitä, että
BLDC moottorissa kestomagneetti sijaitsee pyörivässä osassa, eli roottorissa,
ja käämi, jonka avulla moottori pyörii, sijaitsee paikallaan pysyvässä
osassa, eli staattorissa. Koska käämi ei liiku, sen läpi kulkevan
virran suunta täytyy vaihtaa elektronisesti, mikä tekee BLDC-moottorin
ohjauksesta harjallista moottoria monimutkaisempaa.~\cite{Gabriel2011}

Lennonohjauspiiri lähettää pulssinleveysmodulaatiosignaalina, yleisemmin PWM
(pulse width modulation), halutun kierrosnopeuden nopeudensäätimelle,
yleisemmin ESC:lle (electronic speed control). ESC-piirien toiminta riippuu
siitä, minkälaista moottoria ne ohjaavat. Kaikki ESC-piirit kuitenkin syöttävät
teholähteestä virtaa moottorille niin, että moottorinnopeus vastaa ESC:n
ohjaussignaalia. BLDC-moottori vaatii ESC:n, koska sen käämejä täytyy kytkeä
päälle ja pois oikeassa tahdissa.~\cite{Gabriel2011}


\subsection{Ohjauspinnat}
Edellä käsiteltyjä moottoreita käytetään kiinteäsiipisissä malleissa vain
työntövoimaa varten. Kiinteäsiipisiä malleja ohjataan siivissä ja perän
siivekkeissä olevien ohjauspintojen avulla. Ohjauspinta tarkoittaa siiven osaa,
joka liikkuu. Ohjauspintojen avulla voidaan hallita siiven yli liikkuvaa
ilmavirtaa, ja näin ohjata siipeen kohdistuvaa voimaa, esimerkiksi kääntymistä
varten.

Työntövoimaa varten käytettävät moottorit eivät sovellu kovin hyvin
ohjauspintojen ohjausta varten. Ohjauspinnoissa tarvitaan tarkkuutta, ja
pyörimisnopeus ei ole niin tärkeää. Siksi servomoottorit soveltuvat tähän
tehtävään hyvin. Servomoottori on sähkömoottori, joka on optimoitu toimimaan
tarkasti ja tasaisesti pienillä nopeuksilla. Se sisältää usein myös anturin,
joka mittaa servon asentoa, mahdollistaen näin tarkan ohjauksen.~\cite{Suh2008}

Tutkimusta on tehty paljon myös niin sanottuun ``mukautuvaan siipiprofiiliin''
(adaptive airfoil), jossa koko siipi, tai entistä isompi osa siitä, mukautuu
tilanteen tai ohjauksen tarpeen mukaan. Näin voidaan parantaa esimerkiksi
dronen ohjattavuutta, ja vähentää sakkauksen riskiä.~\cite{Huang2013}
Mukautuvat siipiprofiilit voivat hyödyntää perinteisiä mekaanisia
aktuaattoreita, mutta myös muistimetalleja (Shape Memory Alloy, SMA) on
tutkittu. SMA:n suurin etu perinteisiin aktuaattoreihin on niiden
keveys.~\cite{Abdullah2010}

\chapter{Kommunikaatiojärjestelmät}
\label{ch:kommunikaatio}

Tässä luvussa käsitellään UAV:n kommunikaatiojärjestelmiä ja niiden
erityispiirteitä ja haasteita.

Harraste- ja kuvauskäytössä olevat dronet ovat yleensä niin lähellä
käyttäjäänsä, että yksinkertainen radiolinkki riittää ohjaukseen. Sotilas- ja
tutkimuskäytössä olevat dronet voivat kuitenkin kulkea pitkiä matkoja, lentää
erittäin korkealla tai toimia toisella puolella maailmaa. Tällöin
kommunikaatiojärjestelmät luonnollisesti monimutkaistuvat.

\section{Kommunikaatio UAV:n ja komentokeskuksen välillä}

Droneja ohjataan yleensä langattomasti radiosignaaleilla. Radiosignaalien
lisäksi dronejen ohjauksessa on myös kokeiltu laseria ja valokuitukaapelia.
Laser-ohjauksen kehitys on kuitenkin suurimmaksi osaksi hylätty johtuen
ilmakehän rajoittamasta kantamasta ja luotettavuudesta.

Myös valokuitukaapeli voi olla paras ratkaisu jossain tilanteissa.  Valokuitu
on materiaalia, joka ohjaa toiseen päähän osoitetun valon ulos toisesta päästä.
Valokuitukaapeli mahdollistaa nopean kommunikaation, kun sen läpi johdetaan
laser-signaali.  Valokuitukaapeli on paras vaihtoehto tehtävissä, joissa
tarvitaan todella nopeaa datan lähetystä, käytännössä murtamatonta
tietoturvallisuutta, ja jossa halutaan estää dronen havaitseminen sen
lähettämien radiosignaalien avulla.~\cite{Austin2010} Tässä tekstissä
käsitellään kuitenkin vain radiosignaalien avulla tapahtuvaa kommunikaatiota.

Radiotaajuudeksi määritellään yleensä taajuudeltaan 3 Hz --- 300 GHz välille
sijoittuva elektronmagneettinen säteily~\cite{Sobot2012}.
Taulukossa~\ref{table:RadioFreqTable} näkyy, miten eri taajuudet luokitellaan
ryhmiin.
\begin{table}[H]
  \caption{Radiotaajuuksien luokituksia~\cite{Sobot2012}}
  \begin{center}
    \includegraphics[width=1.0\textwidth]{radiokaistat}
  \end{center}
\label{table:RadioFreqTable}
\end{table}

Radiotaajuksien käyttö on tarkasti säädeltyä eri valtionvirastojen toimesta,
esimerkiksi Suomessa tästä huolehtii Viestintävirasto.

Radio-ohjauksen ongelmat riippuvat UAV:n käyttötarkoituksesta, ja ne johtuvat
yleensä seuraavista syistä:~\cite{Austin2010}
\begin{itemize}
  \item Radiolähettimen tai -vastaanottimen liian heikko vahvistus
  \item Suuret etäisyydet ja/tai epäedullinen ympäristö (vuoristo, pilvet yms)
  \item Liian pieni tiedonsiirtokapasiteetti (liian kapea kaista ja/tai liian
    hidas yhteys)
  \item Tahallinen tai tahaton signaalin häirintä
\end{itemize}

Näköetäisyydellä käytettävillä droneilla ongelmat johtuvat yleensä liian
heikosta lähettimestä tai vastaanottimesta. Kyseisiä droneja ohjataan usein
harrastekäyttöön tarkoitetuilla, melko pienillä lähettimillä, joiden
lähetysteho on melko rajallinen.~\cite{Austin2010}

Sotilaskäytössä olevia UAV:ta käytetään yleensä suurella etäisyydellä
komentokeskuksesta, jolloin niiden kommunikaatio-ongelmat liittyvät usein
etäisyyden tuomiin haasteisiin. Radiosignaalin kantama riippuu monesta asiasta,
mutta parhaimmillaankin se rajoittuu noin 130 kilometriin. Kyseinen kantama
johtuu maapallon kaarevuudesta; Lähetysasema jää yksinkertaisesti horisontin
peittoon, jos oletetaan, että UAV lentää noin kilomterin
korkeudessa.~\cite{Austin2010} Useimmissa tapauksissa ohjaussignaali
välitetäänkin satellitiin tai toisen dronen kautta.

UAV järjestelmien monimutkaisuuden kasvaessa myös niiden tiedonsiirtotarpeet
kasvavat. Monet järjestelmät lähettävät komentokeskukseen reaaliajassa
korkeatasoista kuvaa ja komentokeskus lähettää myös melko monimutkaisiakin
ohjauskomentoja UAV:lle. Samalla myös siviiliväestön tiedonsiirtotarpeet ja
-käyttö kasvavat, jolloin tiedonsiirtokapasiteetti myös jatkuvasti laskee.
Langattoman tiedonsiirron kasvaessa myös häiriöiden määrä kasvaa, puhumattakaan
tahallisesta häirinnästä.~\cite{Austin2010}

\section{Systeemien välinen kommunikaatio ja UAV:t kommunikaatiolinkkeinä}
Edellä kuvattu kommunikaatio UAV:n ja komentokeskuksen välillä on kuitenkin
yksinkertainen tilanne, joka tulee luultavasti muuttumaan entistä
harvinaisemmaksi. Tulevaisuuden UAV:t keskustelevat komentokeskuksen lisäksi
keskenään ja muiden toimijoiden kanssa. Välillä dronejen tulee myös pystyä
kommunikoimaan eri asevoimien järjestelmien kanssa, jonka takia NATO on luonut
NATO STANAG 4586 standardin, joilla eri maiden järjestelmät voivat kommunikoida
toistensa kanssa ainakin välttävästi.\cite{Austin2010}

Dronejen todellinen potentiaali onkin tilanteessa, jossa on useita droneja,
jotka jakavat anturidataa toistensa ja maanpäällisten yksikköjen ja
komentokeskuksen kanssa. 
Lähteessä~\cite{Hayat2014} on kuvattu
IEEE 802.11 protokollaan perustuva UAV-verkko, johon pystytään joustavasti
lisäämään määrittämätön määrä droneja ja tukiasemia.

Dronejen välisen tiedonsiirron parantuessa avautuu myös mahdollisuus käyttää
niitä telekommunikaatiolinkkeinä. Tässä tehtävässä ne ovat paljon halvempia
kuin satelliitit, ja myös joustavampia; Uuden satelliitin lisääminen verkkoon
on hidasta ja kallista, uuden dronen lisääminen on huomattavasti halvempaa ja
nopeampaa.  Erityisesti köyhät, harvaan asutut ja kriisin alaiset alueet
hyötyisivät niistä.~\cite{Li2010}

\chapter{Yhteenveto}
\label{ch:yhteenveto}

Dronejen elektroniset järjestelmät ovat monimutkainen kokonaisuus. Jokaisen
järjestelmän osan suunnittelu riippuu dronen toiminnallisista vaatimuksista,
muista järjestelmän osista ja tietenkin käytössä olevista resursseista.

Akkujen energiatiheyden paraneminen mahdollistaa yhä pienemmät dronet. Suurempi
energiatiheys tarkoittaa, että dronet pystyvät toimimaan pidempään, tai että ne
voivat olla yhä pienempiä. Myös uusiutuvat energianlähteet, energiansieppaus ja
muut vaihtoehdot, kuten polttokennot, parantavat dronejen toiminta-aikaa,
-kantamaa ja mahdollistavat niiden käytön myös tilanteissa jossa resurssit ovat
vähissä. Kaikki kehitys teholähteiden parissa on tärkeää myös ympäristön
kannalta.

Pienentyneet laskentakomponentit mahdollistavat yhä itsenäisempiä UAV:ta. Myös
niiden hinnan lasku on mahdollistanut halvat dronet harraste- ja
tutkimuskäytössä. Lentoa helpottavat autopilottijärjestelmät mahdollistavat
myös nykyään suositut quad-roottorit, joiden ohjaus ilman autopilottia olisi
todella haastavaa.  Laskentakomponenttien tehosta saadaan nykyaikaisella
tekoälyllä myös yhä enemmän irti. Ilmasta otettuja kuvia voidaan analysoida
konenäön avulla, ja näiden analyysien perusteella voidaan tehdä päätöksiä
koneoppimista hyödyntäen.

Dronen antureiden avulla voidaan kerätä monenlaista dataa. Voidaan ottaa
ilmakuvia viihdekäyttöä varten, käyttää infrapunakuvia peltojen tai eläinten
tarkkailua varten, tai mallintaa ympäristöä LIDAR:n avulla. GPS ja IMU piirit
mahdollistavat dronen paikannuksen ja ohjauksen, vaikka näköyhteyttä ei
olisikaan.

Sähkömoottorit ovat pakollinen osa pieniä droneja, joihin polttomoottori olisi
liian iso. Sähkömoottoreilla tuotetaan työtövoimaa, ja quad-roottoreiden
tapauksessa niitä voi myös käyttää ohjaukseen. Servomoottoreilla voidaan ohjata
kiinteäsiipisten dronejen ohjauspintoja.

Dronen kommunikaatiojärjestelmät mahdollistavat tietenkin niiden
kauko-ohjauksen, mutta niillä voi tehdä myös muuta; Niiden avulla dronet voivat
välittää tietoa, keskustella toistensa kanssa ja muodostaa wlan-verkkoja
alueille joissa niitä tarvitaan, esimerkiksi kriisitapauksissa.

Nykyaikaset dronet ovat seurausta kehityksestä jokaisella dronen elektroniikan
osa-alueella, ja jatkokehitys vaatii myös jatkuvaa kehitystä jokaisella
alueella.



\newpage

% \renewcommand{\bibname}{Bibliography}     % Bilingual babel puts Finnish ``Kirjallisuttaa'' otherwise. Strange...
\renewcommand{\bibname}{Lähteet}         % Set Finnish header, remove this if using English
\addcontentsline{toc}{chapter}{Lähteet}  % Include this in TOC
% \addcontentsline{toc}{chapter}{\bibname}  % Include this in TOC

% \bibliographystyle{IEEEtranS}   % the IEEE's sorted numeric style Tää on
% vähän erilainen kun toi ilman äSsää, mutta referenssit ei oo järjestyksessä.
% Harkinnassa että tää ehkä kantsii sitten ihan lopuks ottaa käyttöön, muttei
% ennen
% \bibliographystyle{IEEEtran}   % the IEEE's sorted numeric style

% \bibliographystyle{oma_IEEEtranS}

% \bibliography{ele-kandi}    % Insert {author,title,year...} info of your reference
\printbibliography{}
\markboth{\bibname}{\bibname} % Set page header



\end{document}
