\chapter{Kommunikaatiojärjestelmät}
\label{ch:kommunikaatio}

Tässä luvussa käsitellään UAV:n kommunikaatiojärjestelmiä ja niiden
erityispiirteitä ja haasteita.

Harraste- ja kuvauskäytössä olevat dronet ovat yleensä niin lähellä
käyttäjäänsä, että yksinkertainen radiolinkki riittää ohjaukseen. Sotilas- ja
tutkimuskäytössä olevat dronet voivat kuitenkin kulkea pitkiä matkoja, lentää
erittäin korkealla tai toimia toisella puolella maailmaa. Tällöin
kommunikaatiojärjestelmät luonnollisesti monimutkaistuvat.

\section{Kommunikaatio UAV:n ja komentokeskuksen välillä}

Droneja ohjataan yleensä langattomasti radiosignaaleilla. Radiosignaalien
lisäksi dronejen ohjauksessa on myös kokeiltu laseria ja valokuitukaapelia.
Laser-ohjauksen kehitys on kuitenkin suurimmaksi osaksi hylätty johtuen
ilmakehän rajoittamasta kantamasta ja luotettavuudesta.

Myös valokuitukaapeli voi olla paras ratkaisu jossain tilanteissa.  Valokuitu
on materiaalia, joka ohjaa toiseen päähän osoitetun valon ulos toisesta päästä.
Valokuitukaapeli mahdollistaa nopean kommunikaation, kun sen läpi johdetaan
laser-signaali.  Valokuitukaapeli on paras vaihtoehto tehtävissä, joissa
tarvitaan todella nopeaa datan lähetystä, käytännössä murtamatonta
tietoturvallisuutta, ja jossa halutaan estää dronen havaitseminen sen
lähettämien radiosignaalien avulla.~\cite{Austin2010} Tässä tekstissä
käsitellään kuitenkin vain radiosignaalien avulla tapahtuvaa kommunikaatiota.

Radiotaajuudeksi määritellään yleensä taajuudeltaan 3 Hz --- 300 GHz välille
sijoittuva elektronmagneettinen säteily~\cite{Sobot2012}.
Taulukossa~\ref{table:RadioFreqTable} näkyy, miten eri taajuudet luokitellaan
ryhmiin.
\begin{table}[H]
  \caption{Radiotaajuuksien luokituksia~\cite{Sobot2012}}
  \begin{center}
    \includegraphics[width=1.0\textwidth]{radiokaistat}
  \end{center}
\label{table:RadioFreqTable}
\end{table}

Radiotaajuksien käyttö on tarkasti säädeltyä eri valtionvirastojen toimesta,
esimerkiksi Suomessa tästä huolehtii Viestintävirasto.

Radio-ohjauksen ongelmat riippuvat UAV:n käyttötarkoituksesta, ja ne johtuvat
yleensä seuraavista syistä:~\cite{Austin2010}
\begin{itemize}
  \item Radiolähettimen tai -vastaanottimen liian heikko vahvistus
  \item Suuret etäisyydet ja/tai epäedullinen ympäristö (vuoristo, pilvet yms)
  \item Liian pieni tiedonsiirtokapasiteetti (liian kapea kaista ja/tai liian
    hidas yhteys)
  \item Tahallinen tai tahaton signaalin häirintä
\end{itemize}

Näköetäisyydellä käytettävillä droneilla ongelmat johtuvat yleensä liian
heikosta lähettimestä tai vastaanottimesta. Kyseisiä droneja ohjataan usein
harrastekäyttöön tarkoitetuilla, melko pienillä lähettimillä, joiden
lähetysteho on melko rajallinen.~\cite{Austin2010}

Sotilaskäytössä olevia UAV:ta käytetään yleensä suurella etäisyydellä
komentokeskuksesta, jolloin niiden kommunikaatio-ongelmat liittyvät usein
etäisyyden tuomiin haasteisiin. Radiosignaalin kantama riippuu monesta asiasta,
mutta parhaimmillaankin se rajoittuu noin 130 kilometriin. Kyseinen kantama
johtuu maapallon kaarevuudesta; Lähetysasema jää yksinkertaisesti horisontin
peittoon, jos oletetaan, että UAV lentää noin kilomterin
korkeudessa.~\cite{Austin2010} Useimmissa tapauksissa ohjaussignaali
välitetäänkin satellitiin tai toisen dronen kautta.

UAV järjestelmien monimutkaisuuden kasvaessa myös niiden tiedonsiirtotarpeet
kasvavat. Monet järjestelmät lähettävät komentokeskukseen reaaliajassa
korkeatasoista kuvaa ja komentokeskus lähettää myös melko monimutkaisiakin
ohjauskomentoja UAV:lle. Samalla myös siviiliväestön tiedonsiirtotarpeet ja
-käyttö kasvavat, jolloin tiedonsiirtokapasiteetti myös jatkuvasti laskee.
Langattoman tiedonsiirron kasvaessa myös häiriöiden määrä kasvaa, puhumattakaan
tahallisesta häirinnästä.~\cite{Austin2010}

\section{Systeemien välinen kommunikaatio ja UAV:t kommunikaatiolinkkeinä}
Edellä kuvattu kommunikaatio UAV:n ja komentokeskuksen välillä on kuitenkin
yksinkertainen tilanne, joka tulee luultavasti muuttumaan entistä
harvinaisemmaksi. Tulevaisuuden UAV:t keskustelevat komentokeskuksen lisäksi
keskenään ja muiden toimijoiden kanssa. Välillä dronejen tulee myös pystyä
kommunikoimaan eri asevoimien järjestelmien kanssa, jonka takia NATO on luonut
NATO STANAG 4586 standardin, joilla eri maiden järjestelmät voivat kommunikoida
toistensa kanssa ainakin välttävästi.\cite{Austin2010}

Dronejen todellinen potentiaali onkin tilanteessa, jossa on useita droneja,
jotka jakavat anturidataa toistensa ja maanpäällisten yksikköjen ja
komentokeskuksen kanssa. 
Lähteessä~\cite{Hayat2014} on kuvattu
IEEE 802.11 protokollaan perustuva UAV-verkko, johon pystytään joustavasti
lisäämään määrittämätön määrä droneja ja tukiasemia.

Dronejen välisen tiedonsiirron parantuessa avautuu myös mahdollisuus käyttää
niitä telekommunikaatiolinkkeinä. Tässä tehtävässä ne ovat paljon halvempia
kuin satelliitit, ja myös joustavampia; Uuden satelliitin lisääminen verkkoon
on hidasta ja kallista, uuden dronen lisääminen on huomattavasti halvempaa ja
nopeampaa.  Erityisesti köyhät, harvaan asutut ja kriisin alaiset alueet
hyötyisivät niistä.~\cite{Li2010}
