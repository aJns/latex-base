\chapter{Sulautetut järjestelmät}
\label{ch:sulautetut}

Sulautetuksi järjestelmäksi kutsutaan yleensä tietokonejärjestelmiä, jotka on
suunniteltu rajattua tarkoitusta varten, ja jotka ovat osa jotain isompaa
järjestelmää. Nykyään kyseinen määritelmä ei ole täysin yksiselitteinen.
Perinteistä puhelinta voidaan pitää sulautettuna järjestelmänä, mutta nykyiset
älypuhelimet muistuttavat jo enemmän yleiskäyttöisiä tietokoneita.
Sulautetuilla järjestelmillä on kuitenkin muutamia yhteisiä piirteitä, jotka
erottavat ne yleiskäyttöisistä tietokoneista.~\cite{Noergaard2005}
\begin{itemize}
  \item Niiden fyysinen koko ja/tai laskentateho on selkeästi rajallisempi.
  \item Ne on suunniteltu rajattua tarkoitusta varten.
  \item Niillä on reaaliaikaisuus-vaatimuksia tai niiden toiminnan tulee olla
    muuten luotettavampaa.
\end{itemize}

UAV:n tietokonejärjestelmät ovat melko selkeästi sulautettuja järjestelmiä.
Niiden toiminnalla on tietty tarkoitus, niiden koko ja teho on rajattu
tiettyihin vaatimuksiin, ja niiden tulee ehdottomasti toimia reaaliaikaisesti
ja luotettavasti.

Tässä luvussa kerrotaan dronejen sulautetuista järjestelmistä, ja dronen niihin
kohdistamista erityisvaatimuksista. Ensin perehdytään sulautettujen
järjestelmien arkkitehtuuriin, eli järjestelmien elektroniikan ja ohjelmiston
suunnitteluun droneille. Esille otetaan myös muutamia kehitysalustoja ja
kirjastoja, jotka ovat tarkoitettu dronen sulautettujen järjestelmien
rakentamiseen. Sen jälkeen käydään tarkemmin läpi näiltä järjestelmiltä
vaadittavaa toiminnallisuutta.

\section{Elektroniikka}
Pienemmät UAV:t, joita käytetään harrastuksessa tai tutkimuksessa, ovat paljon
yksinkertaisempia. Harrastedroneille on tarjolla todella paljon erilaisia
lennonohjauslevyjä, kuten esimerkiksi PyroDrone~\cite{PyroDroneFC}. Kyseinen
levy on melko tyypillinen harrastedronen lennonohjauslevy. Sen keskiössä on
32-bittinen STM32F405 ARM-arkkitehtuuriin perustuva prosessori, jonka
toiminnallisuutta on jatkettu muun muassa flash-muistilla, monenlaisilla
liitännöillä ja MPU6000 inertiamittauspiirillä.

Linkboard~\cite{Wzorek2017} on pienikokoisille UAV:lle suunniteltu alusta,
jonka laitteisto on muunneltavissa riippuen siitä, kuinka paljon laskentatehoa
tarvitaan. Minimissään laittesto pystyy lennonohjaukseen ja anturidatan
käsittelyyn. Siihen voi kuitenkin lisätä kaksi Gumstix lisälevyä. Lisälevyt
ovat identtisiä, ja sisältävät yhden ARM Cortex-A8 prosessorin ja digitaalisen
signaalinkäsittelypiirin (digital signal processor, DSP). Lisälevyjen avulla
Linkboard pystyy korkeatasoiseen lennon suunnitteluun ja datan käsittelyyn.

\section{Ohjelmisto}
Koska kyseessä on sulautettu järjestelmä, ohjelmiston toteutus riippuu pitkälti
käytetystä elektroniikasta. Yksinkertaisissa malleissa ei välttämättä käytetä
käyttöjärjestelmää, vaan mikrokontrolleri suorittaa yhtä ohjelmaa, joka ohjaa
dronea. Toiminnallisuuden kasvaessa ohjelmisto voi kasvaa hallitsemattomaksi,
jolloin on hyödyllistä käyttää olemassaolevia kehitysalustoja.

Yksi esimerkki tälläisestä kehitysalustasta on Robot Operating System (ROS),
joka kerää yhteen työkaluja ja kirjastoja helpottamaan robottia ja
miehittämättömien ajoneuvojen ohjelmistokehitystä~\cite{RobotOperatingSystem}.
Erityisesti droneille on olemassa myös flyt\@{OS}. Flyt\@{OS} toimittaa
pitkälti samaa virkaa kuin ROS, mutta se keskittyy erityisesti
drone-sovelluksiin ja niiden vaatimuksiin.~\cite{flytOS}

Kauko-ohjattaville harrastelennokeille ja multiroottoreille on olemassa
lukuisia lennonohjausohjelmistoja. Yksi esimerkki on Cleanflight. Cleanflight
on suhteellisen helposti konfiguroitava ohjelmisto, joka muun muassa helpottaa
lennokin ohjausta PID-säädöin ja mahdollistaa vaikkapa LED-nauhojen
kauko-ohjauksen.~\cite{Cleanflight}


\section{Toiminnallisuus}

Kuvassa~\ref{fig:uav_ohjaus_tasot} on periaatekuva UAV:n ohjauksen tasoista.
Tasot on kuvattu laskevan abstraktion mukaan; Ylimpänä on melko käsitteellinen
``Tehtävä'', alimpana todella yksityiskohtaista hallintaa vaativat
aktuaattorit. Katkoviivalla on rajattu tasot, jotka voivat kuulua UAV:n
sulautettujen järjestelmien toiminnallisuuteen.
\begin{figure}[H]
  \begin{center}
    \includegraphics[width=1.0\textwidth]{uav_ohjaus_tasot}
  \end{center}
  \caption{Periaatekuva UAV:n ohjauksen tasoista.}
\label{fig:uav_ohjaus_tasot}
\end{figure}

Kauko-ohjattavan harrastedronen tietokonejärjestelmien toiminnallisuudella ei
ole paljon vaatimuksia. Vaatimuksena on lähinnä helpottaa ohjausta säätämällä
moottoreita.  Mahdollisesti halutaan myös ohjata vaikkapa dronesta löytyviä
LED:ejä.  Itsenäisen, tiedusteluun tai tiedon keräykseen tarkoitetun, UAV:n
ohjelmisto on jo astetta monimutkaisempi. Enää ei riitä vain moottorin säätö,
tarvitaan myös täysin varusteltu autopilotti, ja itsenäisyyden tasosta riippuen
ehkä myös jonkun tason tekoäly, joka päättää mitä UAV seuraavaksi tekee. Jos UAV
lentää matalalla, sen tulee pystyä väistämään maaston piirteitä, ja on myös
mahdollista, että UAV:n halutaan itse pystyvän, jossain määrin, tulkitsemaan
havaintojaan.
Tämän takia tässä tekstissä on eroteltu yksinkertaisempi
autopilotti ja monimutkaisempi tekoäly omiin alakappaleisiinsa.

\subsection{Autopilotti}
Perinteisen autopilotin toiminnallisuus voidaan jakaa kahteen osaan: Lentävän
koneen, esimerkiksi lentokoneen tai dronen, vakautus, ja ohjaus, joka pitää
koneen reitillään. Vakautus pyrkii pitämään koneen mahdollisimman vakaana,
ohjaamalla esimerkiksi jatkuvaa sivutuulta vastaan. Lennonvakausjärjestelmä voi
myös helpottaa ohjausta abstrahoimalla esimerkiksi moottorinnopeuden säädön.
Reitinohjaus taas pyrkii pitämään koneen halutulla reitillä. Modularisuuden
kannalta on hyödyllistä suunnitella vakaus ja ohjaus erillisinä lohkoina, niin
kuin lähteessä~\cite{Capello2017}, jossa on kuvattu autopilotti, jonka
reitinohjausjärjestelmä ohjaa vakautusjärjestelmää.

Harrastedroneissa on yleensä käytössä vain lennonvakausjärjestelmä. Nykyisin
suosittuja multiroottoreita olisi huomattavasti vaikeampi ohjata, jos ohjaaja
joutuisi säätämään suoraan jokaisen moottorin pyörimisnopeutta. Sen sijaan
ohjaaja voi antaa dronelle komennon kallistua vasemmalle. Tämä on mahdollista
lennonohjauslevyllä olevan säädön seurauksena. Harrastedronejen
lennonohjauslevyt käyttävät yleensä PID-säädintä. PID-säädin on suljetun
silmukan säädin, joka tarkoittaa, että säädin huomioi ulostulossaan säätönsä
vaikutuksen. Droneissa tämä tarkoittaa, että se tarkkailee drone korkeuden,
nopeuden ja asennon muutosta asettaessaan moottoreiden
pyörimisnopeuden. Se ottaa antureiden datan huomioon säädössään, jolloin, jos
drone ei kallistu tarpeeksi, se voi lisätä moottoreiden kierroksia, kunnes
kallistuma on tarpeeksi suuri.~\cite{Sebbane2015}

PID-säädin on vanha ja yksinkertainen keksintö, mutta se on edelleen usein
paras ratkaisu käytännön säätöongelmaan. PID-säädin saa nimensä eri
toimintalohkoistaan; P-lohkon ulostulo on halutun arvon ja nykyisen arvon
erotus kerrottuna vakiolla. I-lohko integroi erotusta ajan yli, eli käytännössä
summaa erotuksen arvoja asetetulta ajalta, ja kertoo summan vakiolla. D-lohko
derivoi erotusta, eli laskee, kuinka nopeasti erotus muuttuu, ja kertoo
derivaatan vakiolla. PID-säätimen muutettavia parametreja ovat jokaisen lohkon
kerroin, ja integraalin ja derivaatan aika. Näitä parametreja muuttamalla
säädin voidaan muokata sopivaksi monenlaisiin
säätötehtäviin.~\cite{Visioli2006}

Keskeistä lennonohjauksessa on UAV:n liikeradan mallintaminen ja suunnittelu.
Etenkin kiinteäsiipisen UAV:n on otettava huomioon monimutkaisia
lentodynamiikkoja. Se ei esimerkiksi saa asettaa konetta liian jyrkkään
nousuun, ettei kone sakkaa. Yksinkertaisempi ohjausjärjestelmä saattaa ottaa
huomioon ainoastaan koneen omat ominaisuudet, esimerkiksi sen siipien nosteen
ja moottorien tehon. Monimutkaisemmat järjestelmät voivat ottaa huomioon myös
meteorologista tietoa, esimerkiksi ohittamalla, alueet joilla on kova
vastatuuli.~\cite{Sebbane2015}

\subsection{Tekoäly ja konenäkö}
Jos autopilotin tehtävä on pitää UAV reitillään, UAV:n tekoälyn tulee valita
seurattava reitti. Reitin valintaan vaikuttavat monet seikat, kuten
suoritettava tehtävä, ympäristö, ja käytettävissä olevat anturit ja
aktuaattorit.

Robotiikassa, johon itsenäiset UAV:t lasketaan~\cite{Sebbane2014}, tekoälystä
puhutaan rationaalisena toimijana. Tässä yhteydessä toimija, englanniksi agent,
voi tarkoittaa mitä tahansa objektia, joka havaitsee ympäristöään antureiden
kautta, ja toimii siinä aktuaattoreiden avulla. Rationaalinen tarkoittaa, että
toimija päättää jokaisessa tilanteessa tehdä sen toiminnon, jonka odotettu
seuraus on mahdollisimman hyvä. Toiminnan hyvyyden arviointi on kuitenkin
vaikea ja keskeinen ongelma tekoälyn kehityksessä.~\cite{Russell2013}
Antureita ja aktuaattoreita käsitellään tarkemmin kappaleissa~\ref{ch:anturit}
ja~\ref{ch:aktuaattorit}.

Reitinsuunnitteluun löytyy monia menetelmiä. Käytetty menetelmä riippuu myös,
minkä tasoisesta reitinsuunnittelusta on kyse. Jos on tarve päästä paikasta A
paikkaan B, ja ympäristö tunnetaan tarkasti, on tarjolla useita
reititysalgoritmeja.~\cite{Sebbane2014} Jos taas tilanne on muuttuva,
esimerkiksi, koska tehtävänä on seurata liikkuavaa kohdetta, tarvitaan
monimutkaisempia ratkaisuja. Artikkelissa~\cite{Ragi2013} on käytetty 
osittain havaittavissa olevaa Markov-päätöksentekoprosessia (partially
observable Markov decision process, POMDP).
Artikkelissa tarkoituksena on ohjata useampaa UAV:ta muuttuvassa ympäristössä,
joiden tulee seurata kohdettaan, kuitenkin samalla pitäen etäisyyttä uhkiin.

Reitinsuunnittelua varten UAV:n täytyy myös pystyä mallintamaan ympäristöään.
Suoraan antureista saatu data, kuten kuvat tai LIDAR:sta saatu etäisyysdata,
täytyy käsitellä konenäön avulla.
Konenäkö on tekoälyn osa-alue, jonka tarkoituksena on mahdollistaa ympäristön
havainnointi koneille yhtä tehokkaasti kuin ihmiselle. Konenäössäkin on
useampia tasoja. Alemmalla tasolla käsitellään kuvia, jotta halutut asiat
korostuisivat. Voidaan esimerkiksi korostaa reunoja. Korkeammalla tasolla
käytetään koneoppimisen metodeja, ja koitetaan esimerkiksi tunnistaa, mitä kuva
esittää.~\cite{Davies2012}

Konenäöllä voi olla erilaisia tavoitteita.
Sillä voidaan havainnoida ympäristöä navigointia varten, ja tunnistaa esteitä.
Artikkelissa~\cite{Subramanian2006} on kehitetty järjestelmä sitruspuiden
lomassa ajavan itsenäisen robotin ohjausta varten. Kameralla otettuja kuvia
analysoidaan konenäköjärjestelmällä, joka luo mallin ympäristöstä, jonka avulla
robotti voi ohjata itseään ilman tarkkaa GPS-signaalia.
Konenäköä voi myös käyttää pelastustoimissa ihmisen löytämiseen
maastossa~\cite{Rudol2008}, tai villieläinkantojen
tarkkailuun~\cite{Gonzalez2016}.

