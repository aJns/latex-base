\chapter{Aktuaattorit}
\label{ch:aktuaattorit}

Tässä luvussa käsitellään UAV:n liikkumiseen tarvittavia aktuaattoreita, eli
pääasiassa sähkömoottoreita. Droneissa voi olla muitakin aktuaattoreita.
Sotilasmallit saattavat kantaa aseita, joiden laukaisuun tarvitaan
aktuaattoreita. Tutkimus- tai kuljetuskäyttöön tarkoitetuissa malleissa voi
olla tavaroiden kuljetukseen, tai tutkimuslaitteiden keräämiseen ja
purkamiseen, tarkoitettuja aktuaattoreita. Viihdemalleissa voi olla valoja tai
kaiuttimia.

Näitä aktuaattoreita on kuitenkin niin monenlaisia, että niiden käsittely ei
ole järkevää. Kaikissa droneissa on kuitenkin liikumiseen tarkoitettuja
aktuaattoreita, ja monet muut aktuaattorit, kuten kourat ja asejärjestelmät,
käyttävät myös sähkömoottoreita, joita tässä käsitellään.

Ensin tuodaan esille erilaisia dronekokoonpanoja, jonka jälkeen perehdytään
niiden käyttämiin ohjausaktuaattoreihin ja niiden eroihin.

\section{Dronekokoonpanot}

Kuten miehitetyt lentokoneet, myös UAV:t ja dronet voidaan jakaa
kiinteäsiipisiin ja roottorilentokoneisiin, riippuen siitä, millä tavalla ne
pysyvät ilmassa. Kiinteäsiipisten mallien tulee liikkua tarpeeksi nopeasti,
jotta niiden siivet pystyvät luomaan nostetta. Roottorilentokoneet taas pysyvät
ilmassa vain propellien luoman nosteen avulla. Roottorilentokoneiden reititys
on yksinkertaisempaa, ja ne ovat joustavampia, koska niiden ei tarvitse pysyä
liikkeessä pysyäkseen ilmassa. Kiinteäsiipiset mallit taas ovat
energiatehokkaampia, ja pystyvät kuljettamaan painavampia lasteja, koska
moottorin ei tarvitse yksin huolehtia nosteesta.~\cite{Austin2010}

On olemassa myös ratkaisuja, jotka eivät sovi tähän jakoon, kuten esimerkiksi
ornithoptereita~\cite{Austin2010}, mutta suurimmalle osasta droneista kyseinen
jako toimii hyvin.

Roottorilentokoneita on monenlaisia. Pääroottoreita voi olla yksi, kuten
perinteisessä helikopterissa. Tällöin tarvitaan myös toissijainen roottori
vastustamaan pääroottorin luomaa kierremomenttia. Vaihtoehtoisia malleja ovat
koaksiaaliroottori, jossa on kaksi pääroottoria päällekkäin, pyörien
vastakkaisiin suuntiin, näin kumoten toisen aiheuttaman kierremomentin;
Tandemroottori, jossa on kaksi pääroottoria eri kohdissa dronea; Ja
quad-roottori, joka on varsinkin harrastedroneissa suosittu konfiguraatio.
Quad-roottorin etu harrastedroneissa on se, että dronen hallintaan riittää
moottoreiden nopeuden säätö. Muissa konfiguraatiossa täytyy pystyä hallitsemaan
roottorin siivekkeiden kulmaa, joka lisää monimutkaisuutta ja
hintaa.~\cite{Austin2010}

% \TODO{paulos tähän}
% \cite{Paulos2013}
% \cite{Paulos2015}

Kiinteäsiipisissä droneissa moottori on yleensä takana. Näin dronen etuosaan on
mahdollista sijoittaa esimerkiksi antureita, jossa niillä on parempi näkyvyys.
Kiinteäsiipisten konfiguraatioiden isoin ero tulee perän sijainnista. Se voi
olla osa runkoa, kuten useimmissa matkustajalentokoneissa.  Se voi olla myös
kiinni puomeilla, tai se voi puuttua kokonaan, jolloin kyse on lentävästä
siivestä.~\cite{Austin2010}


\section{Sähkömoottorit}
Pienemmissä droneissa käytetään yleensä sähkömoottoreita, koska ne
ovat polttomoottoreita kevyempiä, mutta kuitenkin tarpeeksi
tehokkaita~\cite{Austin2010}. Yksinkertainen, esimerkiksi harrastedronessa
esiintyvä, moottorijärjestelmä koostuu teholähteestä, nopeudensäätimestä,
sähkömoottorista ja propellerista~\cite{Gabriel2011}. Teholähteitä käsiteltiin
jo kappaleessa~\ref{ch:teholahde}, ja propellerin aerodynamiikan käsittely ei
kuulu tämän työn aihepiiriin. Käsiteltäväksi jäävät siis sähkömoottori ja
nopeudensäädin.

Kaikkien sähkömoottoreiden toiminta perustuu sähkövirran aiheuttamaan
magneettikenttään. Kahden magneettikentän kohdatessa syntyy voima, joka pyrkii
asettamaan magneettikentät samansuuntaisiksi. Jos siis on olemassa esimerkiksi
kestomagneetilla luotu magneettikenttä, ja siihen viedään johto, jonka läpi
virtaa sähköä, magneettinen voima pyrkii asettamaan johdon ja kestomagneetin
niin, että niiden magneettikentät ovat samansuuntaiset. Jos tässä tilanteessa
kestomagneetin liike estetään, vain johto muuttaa asentoaan. Jos johdon läpi
kulkevan virran suuntaa vaihdellaan jaksottaisesti, voidaan saada aikaan
pyörimisliike.~\cite{Gottlieb1997}

Kuvassa~\ref{fig:dc-motor} on yksinkertaistettu kuva tasavirtamoottorista.
Kuvassa nähdään kestomagneetin navat, N ja S, käämi, eli johto, jonka läpi
kulkee virta, teholähde ja harjat. Harjat ovat kuvassa teholähteestä lähtevät
kappaleet, jotka osuvat käämiin. Silmukan pyöriessä harjojen kontaktipinta
vaihtuu, jolloin silmukan läpi virtaavan sähkön suunta vaihtuu.  Ilman tätä
virran suunnan vaihtumista silmukka ei pyörisi.~\cite{Gottlieb1997}
\begin{figure}[H]
  \begin{center}
    \includegraphics[width=1.0\textwidth]{dc-motor}
  \end{center}
  \caption{Yksinkertaistettu kuva tasavirtamoottorista.~\cite{Gottlieb1997}}
\label{fig:dc-motor}
\end{figure}

Droneissa käytetyt sähkömoottorit ovat yleensä harjattomia
tasavirtamoottoreita. Harjattomat tasavirtamoottorit, lyhennettynä BLDC
(brushless direct current) moottorit, ovat suosittuja droneissa, koska niiden
ominaisuudet vastaavat hyvin dronen moottorille asettamia
vaatimuksia.~\cite{Gabriel2011}

Näitä ominaisuuksia BLDC-moottoreissa ovat~\cite{Yedamale2003}
\begin{itemize}
  \item Parempi pyörimisnopeus/vääntömomentti suhde. Vääntömomentti ei vähene
    nopeuden kasvaessa.

  \item Parempi hyötysuhde. Jännite ei laske harjojen yli.

  \item Suurempi lähtöteho kokoonsa nähden.

  \item Suurempi nopeusalue. Harjat eivät rajoita nopeutta.

  \item Aiheuttavat vähemmän elektromagneettista häiriötä.

\end{itemize}

Harjojen puute on myös itsessään etu, sillä harjat kuluvat nopeasti, joten
harjattoman moottorin elinikä on pidempi. Harjattomuus johtuu siitä, että
BLDC moottorissa kestomagneetti sijaitsee pyörivässä osassa, eli roottorissa,
ja käämi, jonka avulla moottori pyörii, sijaitsee paikallaan pysyvässä
osassa, eli staattorissa. Koska käämi ei liiku, sen läpi kulkevan
virran suunta täytyy vaihtaa elektronisesti, mikä tekee BLDC-moottorin
ohjauksesta harjallista moottoria monimutkaisempaa.~\cite{Gabriel2011}

Lennonohjauspiiri lähettää pulssinleveysmodulaatiosignaalina, yleisemmin PWM
(pulse width modulation), halutun kierrosnopeuden nopeudensäätimelle,
yleisemmin ESC:lle (electronic speed control). ESC-piirien toiminta riippuu
siitä, minkälaista moottoria ne ohjaavat. Kaikki ESC-piirit kuitenkin syöttävät
teholähteestä virtaa moottorille niin, että moottorinnopeus vastaa ESC:n
ohjaussignaalia. BLDC-moottori vaatii ESC:n, koska sen käämejä täytyy kytkeä
päälle ja pois oikeassa tahdissa.~\cite{Gabriel2011}


\subsection{Ohjauspinnat}
Edellä käsiteltyjä moottoreita käytetään kiinteäsiipisissä malleissa vain
työntövoimaa varten. Kiinteäsiipisiä malleja ohjataan siivissä ja perän
siivekkeissä olevien ohjauspintojen avulla. Ohjauspinta tarkoittaa siiven osaa,
joka liikkuu. Ohjauspintojen avulla voidaan hallita siiven yli liikkuvaa
ilmavirtaa, ja näin ohjata siipeen kohdistuvaa voimaa, esimerkiksi kääntymistä
varten.

Työntövoimaa varten käytettävät moottorit eivät sovellu kovin hyvin
ohjauspintojen ohjausta varten. Ohjauspinnoissa tarvitaan tarkkuutta, ja
pyörimisnopeus ei ole niin tärkeää. Siksi servomoottorit soveltuvat tähän
tehtävään hyvin. Servomoottori on sähkömoottori, joka on optimoitu toimimaan
tarkasti ja tasaisesti pienillä nopeuksilla. Se sisältää usein myös anturin,
joka mittaa servon asentoa, mahdollistaen näin tarkan ohjauksen.~\cite{Suh2008}

Tutkimusta on tehty paljon myös niin sanottuun ``mukautuvaan siipiprofiiliin''
(adaptive airfoil), jossa koko siipi, tai entistä isompi osa siitä, mukautuu
tilanteen tai ohjauksen tarpeen mukaan. Näin voidaan parantaa esimerkiksi
dronen ohjattavuutta, ja vähentää sakkauksen riskiä.~\cite{Huang2013}
Mukautuvat siipiprofiilit voivat hyödyntää perinteisiä mekaanisia
aktuaattoreita, mutta myös muistimetalleja (Shape Memory Alloy, SMA) on
tutkittu. SMA:n suurin etu perinteisiin aktuaattoreihin on niiden
keveys.~\cite{Abdullah2010}
