Dronet, tai UAV:t (unmanned aerial vehicle), suomeksi miehittämättömät
ilma-alukset, ovat kehittyvä ja kasvava tekniikan alue. Niissä käytettävän
teknologian --- varsinkin tekoälyn --- kehittyessä myös niiden mahdolliset
sovellukset lisääntyvät. Vieläkin niiden merkittävimmät sovelluskohteet ovat
sotilastoiminnassa, mutta onneksi niiden käyttö tutkimuksessa ja
humanitäärisissä tehtävissä on lisääntymässä. Jälkimmäisissä sovelluksissa
dronejen hinnan lasku ja niiden kehittyvä itsenäisyys ovat isoja tekijöitä.

Droneja on myös monenlaisia ja monentasoisia. Harrastekäytössä oleva drone voi
mahtua kämmenelle, pystyy lentämään 15 minuuttia, ja on maksanut alle 50 euroa.
Sotilaskäytössä oleva drone voi olla suurempi kuin lentokone, lentää
vuorokausien ajan monen sadan kilometrin toimintasäteellä, ja on maksanut
vähintään yli miljoona dollaria. Luonnollisesti näiden kahden dronen
vaatimukset ja suunnittelu myös eroavat toisistaan huomattavasti.

Tämän työn tarkoitus on tarkastella dronejen toimintaa mahdollistavaa
teknologiaa, erityisesti niissä olevaa elektroniikkaa. Työssä käsitellään
dronen teholähteitä, sulautettuja järjestelmiä, ohjelmistoa, tekoälyä,
antureita, moottoreita ja kommunikaatiojärjestelmiä. Työssä tarkastellaan myös
tulevaisuudennäkymiä.
